\documentclass[12pt,reqno]{amsart}

\usepackage{graphicx}
\graphicspath{ {./images/} }

\usepackage{float}
\usepackage{amsfonts,amssymb,amsmath,amsopn,amsthm,graphicx}
\usepackage{amsxtra, mathrsfs}
\usepackage{colordvi}
\usepackage[usenames,dvipsnames]{color}
\usepackage{amsfonts,amssymb,amsbsy,amsmath,amsthm,dsfont}
%%%%%%%%%%%
\usepackage{xspace}
\usepackage{pgfplots}
\usepackage{lipsum}
\usepackage[many]{tcolorbox}
\usetikzlibrary{decorations.pathreplacing}




\linespread{1.1} \numberwithin{equation}{section}


\newcommand\red[1]{\textcolor{red}{#1}}
\newcommand{\Gluck}{Gl\H{u}ck\xspace}


\newtcolorbox{leftbrace}{%
	enhanced jigsaw, 
	breakable, % allow page breaks
	frame hidden, % hide the default frame
	overlay={%
		\draw [
		decoration={brace,amplitude=0.5em},
		decorate,
		ultra thick,
		]
		% right line
		(frame.south west)--(frame.north west);
	},
	% paragraph skips obeyed within tcolorbox
	parbox=false,
}


%%%%%%%%%
\usepackage{amssymb,amsmath,graphicx,amsthm,a4wide,wrapfig,caption,subcaption,epstopdf}
%\usepackage{amsmath,amssymb,amsfonts,amscd,hyperref,color}
\usepackage[latin1]{inputenc}
%\usepackage[mtoscr]{mtpro2}
\usepackage{enumitem}

%\usepackage[abs]{overpic}
%\usepackage{graphicx}
%\usepackage{verbatim}
%\usepackage{amscd}
\usepackage{verbatim}
%%%%%%%%%%%%%%%%%%%%%%%%%%
%%%%labels%%%%%%%%
%\usepackage[notref,notcite]{showkeys}
%%%%%%%%%%%%%%%




\DeclareRobustCommand{\rchi}{{\mathpalette\irchi\relax}}
\newcommand{\irchi}[2]{\raisebox{\depth}{$#1\chi$}} % inner command, used by \rch
\usepackage{lineno}
\usepackage{textcomp}
\usepackage{mathtools,hyperref}
\usepackage{cleveref}
%\hyperref[label_name]{''link text''}

%\usepackage[utf8]{inputenc}
%\usepackage[english]{babel}
\usepackage{setspace,esint}
%\usepackage{graphicx}
%\usepackage{mathtools}
%\usepackage{caption}
%\usepackage{subcaption}
%\graphicspath{{photos/}}
%\usepackage[usenames, dvipsnames]{color}
%\usepackage{}
%\usepackage{enumitem}
%\newcommand\mysymbol[3]{\protected\gdef#1{#2}%	\item[$#2$]#3}










%%%%%%%%%%%%%%%

%\linespread{1.1} \numberwithin{equation}{section}
%\setlength{\voffset}{-.7truein}
%\setlength{\textheight}{8.8truein}
%\setlength{\textwidth}{6.1truein}
%\setlength{\hoffset}{-.7truein}


%%%%%
\newcommand\blue[1]{\textcolor{blue}{#1}}
%%%%%%%%%%%%%%%%%

\newcommand{\Hmm}[1]{\leavevmode{\marginpar{\tiny%
			$\hbox to 0mm{\hspace*{-0.5mm}$\leftarrow$\hss}%
			\vcenter{\vrule depth 0.1mm height 0.1mm width \the\marginparwidth}%
			\hbox to
			0mm{\hss$\rightarrow$\hspace*{-0.5mm}}$\\\relax\raggedright #1}}}
%%%%%%%%%%%%%%%%%%%%
\newcommand{\wc}{{\overline{\Omega}}}
\newcommand{\bb}{{\mathbf{\bar {b}}}}
\newcommand{\bt}{{\mathbf{\tilde{b}}}}
\newcommand{\Gwb}{{\overline{\Omega}}}
\newcommand{\Hd}{\,H^1_{\Pw_{\mathrm{Dir}}}(\Gw)}

%%%%%%%%%%%%%%%%%%%%%%%
\newcommand{\im}{\mbox{\upshape Im\ }}
\newcommand{\re}{\mbox{\upshape Re\ }}
\newcommand{\tr}{\mbox{\upshape Tr }}
\newcommand{\Pw}{\partial \Omega}

\newcommand{\A}{\mathbf{A}}
\newcommand{\B}{\mathscr{B}}
\newcommand{\ball}{B}
\newcommand{\C}{\mathbb{C}}
\newcommand{\cl}{\mathrm{cl}}
\newcommand{\D}{{\rm D}}
\newcommand{\rr}{{\rm R}}
\newcommand{\Q}{\mathcal{Q}}
\newcommand{\eps}{\varepsilon}
\newcommand{\I}{\mathbb{I}}
\newcommand{\pd}{\partial}
\newcommand{\F}{\mathcal{F}}
\newcommand{\h}{\mathcal{H}}
\newcommand{\K}{\mathcal{K}}
\newcommand{\loc}{{\rm loc}}
\newcommand{\mg}{\mathrm{mag}}
\newcommand{\N}{\mathbb{N}}
\newcommand{\ope}{\mathrm{op}}
\newcommand{\R}{\mathbb{R}}
\newcommand{\sph}{\mathcal{L}}
\newcommand{\s}{\mathcal{S}}
\newcommand{\T}{\mathbb{T}}
\newcommand{\Z}{\mathbb{Z}}
\newcommand{\W}{\mathscr{W}}
\newcommand{\id}{\mathds{1}}
\newcommand{\HD}{H^{1}_{\partial \Omega_{D}}(\Gw)}
\newcommand{\curl}{{\bf curl}}

\newcommand{\pwd}{\partial\Omega_{\mathrm{Dir}}}
\newcommand{\pwr}{\partial\Omega_{\mathrm{Rob}}}
\newcommand{\wb}{\overline{\Omega}\setminus \partial\Omega_{\mathrm{Dir}} }
\newcommand{\Dir}{\mathrm{Dir}}

\newcommand{\sign}{\mathrm{sign}\,}

\newtheorem{theorem}{Theorem}[section]
\newtheorem{corollary}[theorem]{Corollary}
\newtheorem{cor}[theorem]{Corollary}
\newtheorem{thm}[theorem]{Theorem}
\newtheorem{lem}[theorem]{Lemma}
\newtheorem{lemma}[theorem]{Lemma}
\newtheorem{proposition}[theorem]{Proposition}
\newtheorem{example}[theorem]{Example}

\newtheorem{definition}[theorem]{Definition}
\newtheorem{defi}[theorem]{Definition}
\newtheorem{remark}[theorem]{Remark}

\newtheorem{rem}[theorem]{Remark}

\newtheorem{remarks}[theorem]{Remarks}


\theoremstyle{definition}

%\newtheorem{remark}[theorem]{Remark}
\newtheorem{assumption}[theorem]{Assumption}
\newtheorem{assumptions}[theorem]{Assumptions}
\newtheorem{defin}[theorem]{Definition}

%%%%%%%%%%%%%%%%%%%%%
\numberwithin{equation}{section}
\newcommand{\diver}{\mathrm{div}\,}
%\newcommand{\loc}{{\mathrm{loc}}}
\newcommand{\RN}[1]{%
	\textup{\uppercase\expandafter{\romannumeral#1}}%
}
\newcommand{\dx}{\,\mathrm{d}x}
\newcommand{\dy}{\,\mathrm{d}y}
\newcommand{\dz}{\,\mathrm{d}z}
\newcommand{\dt}{\,\mathrm{d}t}
\newcommand{\du}{\,\mathrm{d}u}
\newcommand{\dv}{\,\mathrm{d}v}
\newcommand{\dV}{\,\mathrm{d}V}
\newcommand{\ds}{\,\mathrm{d}s}
\newcommand{\dr}{\,\mathrm{d}r}
\newcommand{\dS}{\,\mathrm{d}S}
\newcommand{\dk}{\,\mathrm{d}k}
\newcommand{\dphi}{\,\mathrm{d}\phi}
\newcommand{\dtau}{\,\mathrm{d}\tau}
\newcommand{\dxi}{\,\mathrm{d}\xi}
\newcommand{\deta}{\,\mathrm{d}\eta}
\newcommand{\dsigma}{\,\mathrm{d}\sigma}
\newcommand{\dtheta}{\,\mathrm{d}\theta}
\newcommand{\dnu}{\,\mathrm{d}\nu}
\newcommand{\dmu}{\,\mathrm{d}\mu}
\newcommand{\drho}{\,\mathrm{d}\rho}
\newcommand{\dvrho}{\,\mathrm{d}\varrho}
\newcommand{\dkappa}{\,\mathrm{d}\kappa}
%%%%%%%%%%%%%%%%%%%%%%%%
\newcommand{\core}{C_0^{\infty}(\Omega)}
\newcommand{\coredir}{C_0^{\infty}(\overline{\Omega} \setminus \pwd)}
\newcommand{\sob}{W^{1,p}(\Omega)}
\newcommand{\sobloc}{W^{1,p}_{\mathrm{loc}}(\Omega)}
\newcommand{\merhav}{{\mathcal D}^{1,p}}
\newcommand{\be}{\begin{equation}}
	\newcommand{\ee}{\end{equation}}
%\newcommand{\mysection}[1]{\section{#1}\setcounter{equation}{0}}
%%%%%%%%%%%%%%%
\newcommand{\bea}{\begin{eqnarray}}
	\newcommand{\eea}{\end{eqnarray}}
\newcommand{\bean}{\begin{eqnarray*}}
	\newcommand{\eean}{\end{eqnarray*}}
\newcommand{\thkl}{\rule[-.5mm]{.3mm}{3mm}}
%%%%%%%%%%%%%%%%%
\newcommand{\Rob}{\mathrm{Rob}}
\newcommand{\Real}{\mathbb{R}}
\newcommand{\real}{\mathbb{R}}
\newcommand{\Nat}{\mathbb{N}}
\newcommand{\ZZ}{\mathbb{Z}}
\newcommand{\Proof}{\mbox{\noindent {\bf Proof} \hspace{2mm}}}
\newcommand{\mbinom}[2]{\left (\!\!{\renewcommand{\arraystretch}{0.5}
		\mbox{$\begin{array}[c]{c}  #1\\ #2  \end{array}$}}\!\! \right )}
\newcommand{\brang}[1]{\langle #1 \rangle}
\newcommand{\vstrut}[1]{\rule{0mm}{#1mm}}
\newcommand{\rec}[1]{\frac{1}{#1}}
\newcommand{\set}[1]{\{#1\}}
\newcommand{\dist}[2]{\mbox{\rm dist}\,(#1,#2)}
\newcommand{\opname}[1]{\mbox{\rm #1}\,}
\newcommand{\supp}{\opname{supp}}
\newcommand{\mb}[1]{\;\mbox{ #1 }\;}
\newcommand{\undersym}[2]
{{\renewcommand{\arraystretch}{0.5}  \mbox{$\begin{array}[t]{c}
				#1\\ #2  \end{array}$}}}
\newlength{\wex}  \newlength{\hex}
\newcommand{\understack}[3]{%
	\settowidth{\wex}{\mbox{$#3$}} \settoheight{\hex}{\mbox{$#1$}}
	\hspace{\wex}  \raisebox{-1.2\hex}{\makebox[-\wex][c]{$#2$}}
	\makebox[\wex][c]{$#1$}   }%
%%Macros for changing font size in math.
\newcommand{\smit}[1]{\mbox{\small \it #1}}% only for letters, numbers
\newcommand{\lgit}[1]{\mbox{\large \it #1}}% only for letters, numbers
\newcommand{\scts}[1]{\scriptstyle #1}
\newcommand{\scss}[1]{\scriptscriptstyle #1}
\newcommand{\txts}[1]{\textstyle #1}
\newcommand{\dsps}[1]{\displaystyle #1}
\newcommand{\ass}[1]{Let Assumptions~\ref{assump1} hold  in a bounded Lipschitz domain $\Gw$}
%%%%%%%%%%%%%%%%%%%%%%%%%%%%%%%Macros for Greek letters.

%%%%%%%%%%%%%%%%%%%%%%%%%%%
\def\ga{\alpha}     \def\gb{\beta}       \def\gg{\gamma}
\def\gc{\chi}       \def\gd{\delta}      \def\ge{\epsilon}
\def \gth{\theta}                         \def\vge{\varepsilon}
\def\gf{\phi}       \def\vgf{\varphi}    \def\gh{\eta}
\def\gi{\iota}      \def\gk{\kappa}      \def\gl{\lambda}
\def\gm{\mu}        \def\gn{\nu}         \def\gp{\pi}
\def\vgp{\varpi}    \def\gr{\rho}        \def\vgr{\varrho}
\def\gs{\sigma}     \def\vgs{\varsigma}  \def\gt{\tau}
\def\gu{\upsilon}   \def\gv{\vartheta}   \def\gw{\omega}
\def\gx{\xi}        \def\gy{\psi}        \def\gz{\zeta}
\def\Gg{\Gamma}     \def\Gd{\Delta}      \def\Gf{\Phi}
\def\Gth{\Theta}
\def\Gl{\Lambda}    \def\Gs{\Sigma}      \def\Gp{\Pi}
\def\Gw{\Omega}     \def\Gx{\Xi}         \def\Gy{\Psi}
%%%%%%%%%%%%%


%%%%%%%%%%%%%%%%%%%%%%%%%%%%%%%%%%%%%%%%%
\begin{document}
	
	
	\title{Fourth Order Accurate in Space and Time Compact scheme for First Order Time Dependent Maxwell Equations}
	
	\author {I. Versano}
	
	\address {School of Mathematical Sciences, Tel-Aviv University, Tel-Aviv 6997801, Israel}
	
	\email {idanversano@tauex.tau.ac.il}
	
	\author {E. Turkel}
	
	\address {School of Mathematical Sciences, Tel-Aviv University, Tel-Aviv 6997801, Israel}
	
	\email {turkel@tauex.tau,ac.il}
	
	\author {S. Tsynkov }
	
	\address {North Carolina State University, Box 8205, Raleigh, NC 27695, USA.}
	
	\email {tsynkov@math.ncsu.edu}
	%
	%\author{
		%	I. Versano {}\thanks{Corresponding author.
			%		School of Mathematical Sciences, Tel-Aviv University, Tel-Aviv 6997801, Israel, E-mail: idanv@campus.technion.ac.il} \qquad
		%	E. Turkel {}\thanks{
			%		School of Mathematical Sciences, Tel-Aviv University, Tel-Aviv 6997801, Israel, E-mail: turkel@tauex.tau,ac.il} \qquad
		%	S. Tsynkov {}\thanks{
			%		Department of Mathematics,  North Carolina State University, Box 8205, Raleigh, NC 27695, USA.
			%		tsynkov@math.ncsu.edu} 
		%}
	
	
	
	
	
	%\fi
	
	%%%%%%%%%%%%%%
	
	
	\maketitle
	\begin{abstract}
		
		
		\medskip
		
		\noindent  2000  \! {\em Mathematics  Subject  Classification.}
		.\\[1mm]
		\noindent {\em Keywords:} 
	\end{abstract}
	
	\section{Introduction}
	
	\section{Maxwell equations}
	The Maxwell equations (without polarization), in first order differential vector form, are given by
	\begin{align}
		\label{eq:maxwellv}
		\frac{\partial \bf{D}}{\partial t} &= \nabla \!\times\! \bf{H} - \bf{J}  & \frac{\partial \bf{B}}{\partial t} &= - \nabla \!\times\! \bf{E} \notag \\
		\nabla \!\cdot\! \bf{B} &= 0 & \nabla \!\cdot\! \bf{D} &= \rho \\
		\bf{B} &= \mu \bf{H} & \bf{D} &= \epsilon \bf{E}. \notag
	\end{align}
	with initial conditions given for $\bf{D}$ and $\bf{E}$ at $t=0$.

We assume the following:
\begin{enumerate}
	\item  The functions $\mu,\varepsilon$ are    positive constants satisfying $\frac{1}{\mu \varepsilon}=c^2$.
	\item  $\Gw=[0,1]^3$, $\vec{\nu} \times {\bf E}=0$ on $\partial \Gw$ (\cite[Section 8]{rolf}).
\end{enumerate}
By letting  $\Delta \tau=c\Delta t$,  $Z=\sqrt{\mu/\varepsilon}$, the rescaled equations reads as follows
%$$
%\frac{\partial {\bf E}}{\partial \tau }=Z(\nabla \times \bf{H}-{\bf J}), \quad 
%\frac{\partial {\bf H}}{\partial \tau }=-\frac{1}{Z}\nabla \times {\bf E}.
%$$
	\begin{align}
	\label{eq:maxwellv}
\frac{\partial {\bf E}}{\partial \tau }&=Z(\nabla \times \bf{H}-{\bf J}) & \frac{\partial {\bf H}}{\partial \tau }&=-\frac{1}{Z}\nabla \times {\bf E} \notag \\
	\nabla \!\cdot\! \bf{B} &= 0 & \nabla \!\cdot\! \bf{D} &= \rho \\
	\bf{B} &= \mu \bf{H} & \bf{D} &= \epsilon \bf{E}. \notag
\end{align}
\begin{rem}
	In Appendix \ref{appendix} we show how our scheme can be extended to the cases $\varepsilon$ and $\mu$ are positive scalar functions which do not depend on $t$.
\end{rem}


	%\begin{remark}
	%	The invariance of the curl operator to rotations imply that
	%our scheme  can be easily generalized to the  case where $\varepsilon$ and $\mu$ are  uniformly   positive definite matrix functions (cf. \cite[Section 8]{rolf}).
	%\end{remark}
	\section{discretization}
	We assume for simplicity that $\Delta x=\Delta y=\Delta z=h$
	and  use the usual notations  $x_i=ih, y_j=jh,z_k=kh$.
	Moreover, we   introduce the following notation:
	$$
	[N]=\{0,1,..,N\},\quad N\in \N.
	$$
	To discretize the equations we introduce a staggered mesh in both space and time as in the Yee scheme \cite{yee}. 
	$\bf{E},\bf{D}$  are evaluated at time $n$ while $\bf{B},\bf{H}$ and $\bf{J}$ are evaluated at time $n\!+\! \frac{1}{2}$.

	
	With this arrangement all space derivatives are spread over a single mesh width and the central time and space derivatives
	are centered at the same point similar to that of the Yee scheme \cite{yee}. 
	We define the following meshes:
	\begin{align*}
		&\Gw^{E_z}:=\{(x_i,y_j,z_{k+\frac{1}{2}}),\quad (i,j,k)\in [N]^2 \times[N-1]\} \\
		& \Gw^{E_y}:=\{(x_i,y_{j+\frac{1}{2}},z_{k}), \quad (i,j,k)\in [N]\times [N-1] \times[N] \}\\
		& \Gw^{E_x}:=\{ (x_{i+\frac{1}{2}},y_{j},z_{k}),\quad (i,j,k) \in [N-1]\times [N]^2 \} \\
		&\Gw^{H_z}:=\{ (x_{i+\frac{1}{2}},y_{j+\frac{1}{2}},z_{k}),\quad (i,j,k)\in [N-1]^2\times[N] \}\\
		&\Gw^{H_y}:=\{ (x_{i+\frac{1}{2}},y_j,z_{k+\frac{1}{2}}),\quad (i,j,k) \in [N-1]\times [N] \times [N-1] \}\\
		& \Gw^{H_x}:=\{ (x_i,y_{j+\frac{1}{2}},z_{k+\frac{1}{2}}), \quad (i,j,k) \in [N]\times [N-1]^2\}.
	\end{align*}
	We denote 
	$$
	\Gw^{{\bf E}}:=\Gw^{E_x}\times \Gw^{E_y}\times \Gw^{E_z}, \qquad 
	\Gw^{\bf H}:=
	\Gw^{H_x}\times \Gw^{H_y}\times \Gw^{H_z}.
	$$
	With this arrangement, the boundary condition 
	$\vec{n}\times {\bf E}=0$ on $\partial \Gw^{{\bf E}}$ readily imply that 
	$E_s=0$ on $\partial \Gw^{E_s}$ for $s=x,y,z$. That is, 
	\begin{align*}
		&
		E_x(x_{i+\frac{1}{2}}0,z_j)=E_x(x_{i+\frac{1}{2}},N-1,z_j)=
		E_x(x_{i+\frac{1}{2}},y_j,0)=E_x(x_{i+\frac{1}{2}},y_j, N-1)=0, \\&
		E_y(0,y_{j+\frac{1}{2}},z_j)=E_y(N-1,y_{j+\frac{1}{2}},z_j)=
		E_y(x_i,y_{j+\frac{1}{2}},0)=E_y(x_i, y_{j+\frac{1}{2}}, N-1)=0, \\&
		E_z(0,y_j,z_{k+\frac{1}{2}})=E_x(N-1,y_j,z_{k+\frac{1}{2}})=
		E_z(x_i,0,z_{k+\frac{1}{2}})=E_z(x_i,N-1,z_{k+\frac{1}{2}})=0. &
	\end{align*} 


	\subsection{Operators}
	We define the following operators for subsequent use.
	For $s=x,y,z$, let $a_s$ be the  finite-difference operator defined  by the left hand side of \eqref{eq:lhs} in Appendix \ref{appendix}.
	
	\begin{enumerate}
		\item $\frac{\Delta \tau }{h}=r$
		\item $\delta_{\tau} U^{n+\frac{1}{2}}:=\frac{U^{n+1}-U^{n}}{\Delta \tau}$
				\item $D_x U(x_i,y_j,z_k):=\frac{U(x_i+h/2,y_j,z_k)-U(x_i-h/2,y_j,z_k)}{h}$
	\item $D_y U(x_i,y_j,z_k):=\frac{U(x_i,y_j+h/2,z_k)-U(x_i-h/2,y_j,z_k)}{h}$
		\item $D_z U(x_i,y_j,z_k):=\frac{U(x_i,y_j,z_k+h/2)-U(x_i-h/2,y_j,z_k)}{h}$

		\item $\delta_x U(x_i,y_j,z_k):=a_x^{-1}\circ D_x$
		\item $\delta_y U(x_i,y_j,z_k):=a_y^{-1}\circ D_y$
		\item 	$\delta_z U(x_i,y_j,z_k):=a_z^{-1}\circ D_z$
		\item $
		L_{\tau}^h(\phi):=-
		\left (
		\frac{D_{xx}\phi}{1+\frac{h^2}{12}D_{xx}}+\frac{D_{yy}\phi}{1+\frac{h^2}{12}D_{yy}}
		+\frac{D_{zz}\phi}{1+\frac{h^2}{12}D_{zz}}\right)
		+\frac{24}{\Delta \tau^2}\phi
		$
		\item
		$D_{ss}\phi=\frac{\phi(s+h)-2\phi(s)+\phi(s-h)}{h^2}$ is standard symmetric second derivative operator for $s=x,y,z$.
	\end{enumerate}
	
	Moreover, we define 
	$$
	P({\bf J}):=Z\frac{24}{\Delta \tau^2}\left ( -{\bf J}^{n+\frac{1}{2}}-\nabla div({\bf J}^{n+\frac{1}{2}})+\frac{\partial^2{\bf J}^{n+\frac{1}{2}}}{\partial t^2}\right ).
	$$
   \subsection{Modified Helmholtz equation}
   In the following sections we will exploit fourth order compact schemes for solving elliptic equations of the form $-\Delta \phi+k^2\phi=F$ in $\Gw^{E_s}$ or $\Gw^{H_s}$, $s=x,y,z$ \cite{singer_turkel}.
   We will use the following scheme:
   $$
   \left (
   \frac{D_{xx}\phi}{1+\frac{h^2}{12}D_{xx}}+\frac{D_{yy}\phi}{1+\frac{h^2}{12}D_{yy}}
   +\frac{D_{zz}\phi}{1+\frac{h^2}{12}D_{zz}}\right)
   +k^2\phi=F.
   $$
   Equivalently,
   \begin{align*}
&
-\left(1+\frac{h^2}{12}D_{yy}\right)
\left(1+\frac{h^2}{12}D_{zz}\right)D_{xx}\phi-
\left(1+\frac{h^2}{12}D_{xx}\right)
\left(1+\frac{h^2}{12}D_{zz}\right)D_{yy}\phi\\&
-\left(1+\frac{h^2}{12}D_{xx}\right)
\left(1+\frac{h^2}{12}D_{yy}\right)D_{zz}\phi+
\left(1+\frac{h^2}{12}D_{xx}\right)
\left(1+\frac{h^2}{12}D_{yy}\right)
\left(1+\frac{h^2}{12}D_{zz}\right)k^2\phi
=\\&
\left(1+\frac{h^2}{12}D_{xx}\right)
\left(1+\frac{h^2}{12}D_{yy}\right)
\left(1+\frac{h^2}{12}D_{zz}\right)F.
\end{align*}
   The latter scheme requires boundary conditions for the inhomogeneous term $F$ and for $\phi$. We denote these boundary conditions as 
   $B_{NH}$ and $B_{C}$ respectively.
   
	\section{The scheme}
	Using  Taylor series and the Maxwell equations to derive the following non-homogeneous elliptic equations for (see Appendix \ref{appendxib})
$	\delta_{\tau} \bf{E}$, and $\delta_{\tau} \bf{H}$.
	:
	$$
	-\Delta\delta_{\tau} \bf{E}^{n+\frac{1}{2}}+\frac{24}{\Delta \tau^2}\delta_{\tau}\bf{E}^{n+\frac{1}{2}}=
	Z\frac{24}{\Delta \tau^2}\nabla\times \bf{H}^{n+\frac{1}{2}}+P(J)^{n+\frac{1}{2}}+
	\mathrm{O(\Delta \tau^4+h^4)},
	$$
		$$
	-\Delta \delta_{\tau}\bf{H}^{n+1}+\frac{24}{\Delta \tau^2}\delta_{\tau}\bf{H}^{n+1}=
	-\frac{1}{Z}\frac{24}{\Delta \tau^2}\nabla\times {\bf E}^{n+1}+
	\mathrm{O(\Delta \tau^4+h^4)}.
	$$
	The resulting scheme reads as follows.
	
	Let $\bf{E}^n_h, \bf{H}^{n+\frac{1}{2}}_h$ (where $h$ denotes discretized function) be given:\\[2mm]
	{\bf step 1}: update $\bf{E}^{n+1}_h$: \\[2mm]
	$$
	\begin{pmatrix}
		E_x^{n+1}\\
		E_y^{n+1}\\
		E_z^{n+1}
	\end{pmatrix}=
	\begin{pmatrix}
		E_x^n+\Delta \tau E_x^{*}\\
		E_y^n+\Delta \tau E_y^{*}\\
		E_y^n+\Delta \tau E_y^{*}
	\end{pmatrix}
	$$
	where 
	$$
	\begin{pmatrix}
		L_{\tau}^{h} & 0&0 \\
		0 & L_{\tau}^{h} &\\
		0&0&L_{\tau}^{h} 
	\end{pmatrix}
	\begin{pmatrix}
		E_x^{*}\\
		E_y^{*} \\
		E_z^{*}
	\end{pmatrix}=Z\frac{24}{\Delta \tau^2}
	\begin{pmatrix}
		0& -\delta_z & \delta _y\\
		\delta_z&0&-\delta_x\\
		-\delta_y&\delta_x&0
	\end{pmatrix}
	\begin{pmatrix}
		H_x^{n+\frac{1}{2}}\\
		H_y^{n+\frac{1}{2}}\\
		H_z^{n+\frac{1}{2}}
	\end{pmatrix}+P({\bf J})
	$$
	in $\Gw^{\bf{E}}$.
	Furthermore,
	$B_{C}=B_{NH}=0$ for $E_s^{*}$ $(s=x,y,z)$.

	
	{\bf step 2}: update $\bf{H}^{n+3/2}_h$: \\[2mm]
	$$
	\begin{pmatrix}
		H_x^{n+3/2}\\
		H_y^{n+3/2}\\
		H_z^{n+3/2}
	\end{pmatrix}=
	\begin{pmatrix}
		H_x^{n+\frac{1}{2}}+\Delta \tau H_x^{*}\\
		H_y^{n+\frac{1}{2}}+\Delta \tau H_y^{*}\\
		H_y^{n+\frac{1}{2}}+\Delta \tau H_y^{*}
	\end{pmatrix}
	$$
	where 
	$$
	\begin{pmatrix}
		L_{\tau}^{h} & 0&0 \\
		0 & L_{\tau}^{h} &\\
		0&0&L_{\tau}^{h} 
	\end{pmatrix}
	\begin{pmatrix}
		H_x^{*}\\
		H_y^{*} \\
		H_z^{*}
	\end{pmatrix}=-\frac{1}{Z}\frac{24}{\Delta \tau^2}
	\begin{pmatrix}
		0& -\delta_z & \delta _y\\
		\delta_z&0&-\delta_x\\
		-\delta_y&\delta_x&0\\
	\end{pmatrix}
	\begin{pmatrix}
		E_x^{n+1}\\
		E_y^{n+1}\\
		E_z^{n+1}
	\end{pmatrix}
	$$
	in $\Gw^{\bf{H}}$. Furthermore,
	$B_{NH}$ equals to the non-homogeneous term which has calculated at the previous step in $\Gw^{\bf{H}}$ including the boundary.
	For $B_{C}$ we use Taylor expansion
	$$
	\frac{\bf{H}^{n+3/2}-\bf{H}^{n+\frac{1}{2}}}{\Delta \tau}=
	\frac{\partial \bf{H}}{\partial t}+O(\Delta \tau ^2)=
	-\frac{1}{Z}	\begin{pmatrix}
		0& -\delta_z & \delta _y\\
		\delta_z&0&-\delta_x\\
		-\delta_y&\delta_x&0\\
	\end{pmatrix}
	\begin{pmatrix}
		E_x^{n+1}\\
		E_y^{n+1}\\
		E_z^{n+1}
	\end{pmatrix}+O(\Delta \tau^2+h^4).
	$$
	Hence 
	$$B_C=
		-\frac{1}{Z}	\begin{pmatrix}
		0& -\delta_z & \delta _y\\
		\delta_z&0&-\delta_x\\
		-\delta_y&\delta_x&0\\
	\end{pmatrix}
	\begin{pmatrix}
		E_x^{n+1}\\
		E_y^{n+1}\\
		E_z^{n+1}
	\end{pmatrix}.
	$$
	
	Approximation of the boundary conditions to order $\Delta \tau ^4$ is also possible as depicted in the example in section \ref{sec:TE}.
	

	

	\section{Energy estimates}
	\begin{lem}[Abel transformation]
		Let $p\geq 1$ be integer and $\{a_k\}_{k=1}^{p}$,$\{b_k\}_{k=1}^{p}$ be two sequences. Then
		$$
		\sum_{k=1}^{p}a_kb_k=a_pB_p-\sum_{k=1}^{p-1}(a_{k+1}-a_k)B_k
		$$
		where 
		$B_k:=\sum_{i=1}^{k}b_i$.
	\end{lem}
\begin{lem}[\cite{Morton}]
Let $\{a_k\}_{k=1}^{p}$,$\{b_k\}_{k=0}^{p}$ be two sequences. Then
$$
\sum_{k=1}^{p}a_k(b_k-b_{k-1})=a_pb_p-a_1b_0-\sum_{k=1}^{p-1}(a_{k+1}-a_k)b_k.
$$
\end{lem}


Let $F,G$ defined in $\Gw^{\bf{E}}$ and $U,V$ defined in 
$\Gw^{\bf{H}}$.
We define the following inner products:
$$
(F,G)_{\bf{E}}:=
h^3\sum_{i=0}^{N-1}\sum_{j=0}^{N-1}\sum_{k=0}^{N-1}
F_{x,i+\frac{1}{2},j,k}G_{x,i+\frac{1}{2},j,k}+
F_{y,i,j+\frac{1}{2},k}G_{y,i,j+\frac{1}{2},k}+F_{z,i,j,k+\frac{1}{2}}G_{z,i,j,k+\frac{1}{2}}
$$
\begin{align*}
	&
	(U,V)_{\bf{H}}:=
	h^3\sum_{i=0}^{N-1}\sum_{j=0}^{N-1}\sum_{k=0}^{N-1}
	F_{x,i,j+\frac{1}{2},k+\frac{1}{2}}G_{x,i,j+\frac{1}{2},k+\frac{1}{2}}+\\&
	F_{y,i+\frac{1}{2},j,k+\frac{1}{2}}G_{y,i+\frac{1}{2},j,k+\frac{1}{2}}+F_{z,i+\frac{1}{2},j+\frac{1}{2},k}G_{z,i+\frac{1}{2},j+\frac{1}{2},k}
\end{align*}
We define the spaces 
$$
V^{\bf{E}}:=\{F \in \Gw^{\bf{E}}:\|F\|_{\bf{E}}<\infty\}, \qquad 
V^{\bf{H}}:=\{G \in \Gw^{\bf{H}}:\|G\|_{\bf{H}}<\infty\}.
$$

We define 
$$
\mathbf{\curl}:V^{\bf{E}}\to V^{\bf{H}}, \quad 
\mathbf{\curl'}:V^{\bf{H}}\to V^{\bf{E}},
$$
by
$$
\mathbf{\curl}( resp. \mathbf{\curl}')=
\begin{pmatrix}
	0& -D_z & D _y\\
	D_z&0&-D_x\\
	-D_y&D_x&0\\
\end{pmatrix}.
$$
\begin{cor}
	For any $\bf{E}_h$ vanishing on $\partial \Gw^{\bf{E}}$ we have 
	$$
	(\mathbf{\curl}(\bf{E}_h),\bf{H}_h)_{\bf{H}}=
		(\mathbf{\curl}'(\bf{H}_h),\bf{E}_h)_{\bf{E}}.
	$$
\end{cor}
Let $P$ be the operator
$$
	P(\phi):=
		-	\left (
	\frac{D_{xx}\phi}{1+\frac{h^2}{12}D_{xx}}+\frac{D_{yy}\phi}{1+\frac{h^2}{12}D_{yy}}
	+\frac{D_{zz}\phi}{1+\frac{h^2}{12}D_{zz}}\right)
	+\frac{24}{\Delta \tau^2}\phi.
	$$
	We define 
	$$
	{\bf P}:V^{\bf{E}}\to V^{\bf{E}}
	$$
	as $$\bf{P}:=
	\begin{pmatrix}
		P & 0&0 \\
		0&P& 0 \\
		0&0 &P
	\end{pmatrix}.
	$$

	Then for all $f,g\in \Gw^{\bf{E}}$:
	\begin{enumerate}
		\item $(\bf{P}g,f)_{\bf{E}}=(\bf{P},g)_{\bf{E}}$.
		\item $({\bf P}f,f)\geq 0$. 
	\end{enumerate}
By a similar way we can define  a symmetric coercieve operator
(which is strictly positive for "good" cfl numbers)
	$$
\bf{P}':V^{\bf{H}}\to V^{\bf{H}}.
$$
\begin{lemma}
	\begin{align*}
		&
\begin{pmatrix}
	0& a_z^{-1}&0 \\
	a_z^{-1}& 0& 0\\
	0&0&0
\end{pmatrix}
\begin{pmatrix}
	0& -D_z&0 \\
	D_z& 0& 0\\
	0&0&0
\end{pmatrix}
{\bf H}^{n+\frac{1}{2}},{\bf E^{n+1}}_{\bf{E}}=\\&
h^3\sum\sum\sum E_{x,i+\frac{1}{2},j,k}
\end{align*}

\end{lemma}
\begin{lem}
	Assume that $U,V$ satisfy the boundary conditions 
	$U_{i+\frac{1}{2},0}=U_{i+\frac{1}{2},N}=V(0,j+\frac{1}{2})=V(N,j+\frac{1}{2})=0$.
	Then
	$$
	\sum_{j=0}^{N-1}W_{i+\frac{1}{2},j+\frac{1}{2}}\delta_yU_{i+\frac{1}{2},j+\frac{1}{2}}=
	-\sum_{j=1}^{N-1}U_{i+\frac{1}{2},j}\delta_y W_{i+\frac{1}{2},j}
	$$
\end{lem}
\begin{lem}
	Let $E_z(i,j)$, $H_x(i,j+\frac{1}{2})$, $H_y(i+\frac{1}{2},j)$.
Then
	\begin{align*}
		&
\sum_{i=0}^{N-1}	\sum_{j=0}^{N-1}a_y^{-1}(i,j)\delta_yE_z(i,j+1/2)H_y(i,j+1/2)=
\sum_{j=0}^{N-1}\delta_yE_z(i,j+1/2)H_y(i,j+1/2)\\&
-\sum_{i=0}^{N-1}	\sum_{j=0}^{N-1}(\delta_{i,j}-a_y^{-1}(i,j))\delta_yE_z(i,j+1/2)H_y(i,j+1/2)
	\end{align*}
\end{lem}
\begin{cor}
	\begin{align*}
		&
		\left(
		\begin{pmatrix}
		0& -\delta_z & \delta _y\\
		\delta_z&0&-\delta_x\\
		-\delta_y&\delta_x&0\\
	\end{pmatrix}
	\begin{pmatrix}
		E_x^{n+1}\\
		E_y^{n+1}\\
		E_z^{n+1}
	\end{pmatrix}, 
\begin{pmatrix}
	H_x^{n+1}\\
	H_y^{n+1}\\
	H_z^{n+1}
	\end{pmatrix}
\right)_{\bf{H}}=\\&
(\mathbf{\curl}(\bf{E}_h),\bf{H}_h)_{\bf{H}}+\mathrm{E}^{1}
	\end{align*}
and $$|\mathrm{E}^{1}|\leq \|I-a^{-1}\|\cdot  \|  {\bf \curl (\bf{E}_h)}\|_{\bf{H}}  \cdot 
\|H\|_{\bf{H}}\leq \frac{2\sqrt{3}}{h}
\|I-a^{-1}\|\cdot  \|  {(\bf{E}_h)}\|_{\bf{E}}  \cdot 
\|H\|_{\bf{H}}
$$
\end{cor}

	We define the following (coercieve )energy function:
$$
\mathrm{Egy}^{n+1}:=({\bf P}\bf{E}^{n+1},\bf{E}^{n+1})_{\bf{E}}+
({\bf P}' {\bf H}^{n+3/2},\bf{H}^{n+3/2})_{\bf{H}}.
$$

	
	
	

%	\subsection{The vaccum case}
%	For constant  $\mu,\varepsilon$ satisfying 
%	$1/(\mu \varepsilon)=c^2$ the scheme can be further simplified to
%	\begin{equation}\label{eq:mu_eps_constant_lap}
%		\delta_t D_z-\frac{c^2\Delta t^2}{24}\Delta ( \delta_t D_z)=
%		\left ( \frac{\partial H_y}{\partial x}-
%		\frac{\partial H_x}{\partial y}-J_z
%		\right)^{n+\frac{1}{2}}+\frac{c^2 \Delta t^2}{24}\left (
%		-
%		\frac{\partial div(\vec{J})}{\partial z}-\frac{\partial^2 J_z}{\partial t^2}
%		\right)^{n+\frac{1}{2}}.
%	\end{equation}
%	Let us define $\Delta \tau=c\Delta t$,  $Z=\sqrt{\mu/\varepsilon}$,
%	$$L_{\tau}(u):=- \frac{\Delta \tau^2}{24}\Delta u+u,$$
%	and finally
%	$$
%	\delta_{\tau} \vec{E}=\frac{\vec{E}^{n+1}-\vec{E}^n}{\Delta \tau}, \qquad
%	\delta_{\tau} \vec{H}=\frac{\vec{H}^{n+\frac{1}{2}}-\vec{H}^{n+3/2}}{\Delta \tau}.
%	$$
%	The full scheme is then given by 
%	\begin{equation}\label{eq:system}
%		\begin{pmatrix}
%			\vec{L}_{\tau} & 0 \\
%			0 & \vec{L}_{\tau}
%		\end{pmatrix}
%		\begin{pmatrix}
%			\delta_{\tau} \vec{E} \\
%			\delta_{\tau} \vec{H}
%		\end{pmatrix}=
%		\begin{pmatrix}
%			\vec{Z}& 0 \\
%			0&	\frac{1}{\vec{Z}}
%		\end{pmatrix}
%		\begin{pmatrix}
%			\nabla \times \vec{H}^{n+\frac{1}{2}} -\vec{J}^{n+\frac{1}{2}}\\
%			-\nabla \times \vec{E}^{n+1}
%		\end{pmatrix}+
%		\frac{\Delta \tau ^2}{24}
%		\begin{pmatrix}
%			\left(-\nabla div(\vec{J})+
%			\frac{\partial^2 \vec{J}}{\partial t^2}\right)^{n+\frac{1}{2}} \\
%			0
%		\end{pmatrix}.
%	\end{equation}
%	The equations in \eqref{eq:system} can be solved in bounded domains to fourth order accuracy using the scheme  in \cite{singer_turkel}. It is important to notice that for bounded domains the boundary conditions for $\delta_t \vec{H}, \delta_t \vec{E}$ has to be approximated to fourth order.
%	
	
	%	+ \frac{\partial^2 {D_x \over \epsilon}}{\partial y \partial t} \right) + O(\Delta t)^4
	%\end{align*}
	%We then substitute the Maxwell equation for $\frac{\partial D}{\partial t}$ and get
	%\begin{equation*}
	%	\frac{\partial B_z}{\partial t} =
	%	\delta_t B_z - \frac{\Delta t^2}{24} \frac{\partial}{\partial t} \left(\frac{\partial^2 {H_z \over \epsilon}}{\partial x^2}  + \frac{\partial^2 {H_z \over \epsilon}}{\partial y^2}
	%	- \frac{\partial^2 {H_x \over \epsilon}}{\partial x \partial z} - \frac{\partial^2 {H_y \over \epsilon}}{\partial y \partial z} + \frac{\partial J_y}{\partial x} - \frac{\partial J_x}{\partial y}\right)
	%	+ O(\Delta t)^4
	%\end{equation*}
	%We now add and subtract $\frac{\partial^2 {H_z \over \epsilon}}{\partial z^2}$ to the terms in the parenthesis and then replace $H$ by $\frac{B}{\mu}$ and then set $c^2 = \frac{1}{\mu \epsilon}$.
	%We note that $c^2$ is constant even when $\epsilon$ and $\mu$ vary.
	%This yields
	%\begin{equation}
	%	\frac{\partial B_z}{\partial t} = \delta_t B_z - \frac{\Delta t^2}{24} \frac{\partial}{\partial t} \left(c^2 \Delta B_z - c^2 \frac{\partial }{\partial z} div \vec{B}
	%	+ \frac{\partial J_y}{\partial x} - \frac{\partial J_x}{\partial y} \right)
	%	+ O(\Delta t)^4
	%	\label{max10}
	%\end{equation}
	%%We further assume that the divergence of both $B$ and $D$ are zero.
	%%If $div(D) = J \neq 0$ then we have an extra term
	%%$\frac{\Delta t^2}{24 c^2} \frac{\partial^2 J}{\partial t \partial z}$,
	%Then \eqref{max10} reduces to
	%\begin{equation}
	%	\frac{\partial B_z}{\partial t} = \delta_t B_z -  \frac{\Delta t^2}{24} \frac{\partial}{\partial t} \left(c^2 \Delta B_z + \frac{\partial J_y}{\partial x} - \frac{\partial J_x}{\partial y} \right)
	%	+ O(\Delta t)^4
	%	\label{max20}
	%\end{equation}
	%Finally, we replace $\frac{\partial \Delta B_z}{\partial t}$ by $\Delta \delta_t B_z +O(\Delta t)^2$.
	%Since it is already multiplied by $(\Delta t)^2$ the total error is $(\Delta t)^4$.
	%So we are left with
	%\begin{equation}
	%	\frac{\partial B_z}{\partial t} = \delta_t B_z - c^2 \frac{\Delta t^2}{24}  \left(\Delta \delta_t B_z + \frac{1}{c^2} \frac{\partial}{\partial t} (\frac{\partial J_y}{\partial x} - \frac{\partial J_x}{\partial y}) \right)
	%	+  O(\Delta t)^4
	%\end{equation}
	% we get
	%\begin{equation}
	%	\label{finalBz}
	%	\delta_t B_z - \frac{c^2 \Delta t^2 }{24}  \Delta (\delta_t B_z)  =
	%	(-\frac{\partial E_y}{\partial x} + \frac{\partial E_x}{\partial y} )^{n+\frac{1}{2}}
	%	+ \frac{\Delta t^2}{24} \frac{\partial}{\partial t} \left( \frac{\partial J_y}{\partial x} - \frac{\partial J_x}{\partial y} \right)
	%	+  O(\Delta t)^4
	%\end{equation}
	%
%	\section{stability analysis}
%	%It seems that $CFL\leq \frac{1}{\sqrt{2}}$ as with Yee.
%	\subsection{The case $\mu, \varepsilon$ are constants and two dimensions}
%	We assume for simplicity that $Z=1$.
%	
%	Let us rewrite  the homogeneous part in   \eqref{eq:system} in two steps:
%	\begin{align*}
%		&
%		\begin{pmatrix}
%			L_{\tau} & 0&0 \\
%			0&L_{\tau} &0 \\
%			0&0&L_{\tau} 
%		\end{pmatrix}
%		\begin{pmatrix}
%			\delta_{\tau} E_x^{n+1} \\
%			\delta_{\tau} E_y^{n+1} \\
%			\delta_{\tau} E_z^{n+1} \\
%		\end{pmatrix}=
%		\frac{1}{h}
%		PD\vec{H}^{n+\frac{1}{2}}
%	\end{align*}
%	\begin{align*}
%		&
%		\begin{pmatrix}
%			L_{\tau} & 0&0 \\
%			0&L_{\tau} &0 \\
%			0&0&L_{\tau}   \\
%		\end{pmatrix}
%		\begin{pmatrix}
%			\delta_{\tau} H_x^{n+3/2} \\
%			\delta_{\tau} H_y^{n+3/2}\\
%			\delta_{\tau} H_z^{n+3/2}
%		\end{pmatrix}=
%		\frac{1}{h}
%		P D^t\vec{E}^{n+1}
%	\end{align*}
%	
%	%
%	%\begin{align*}
%	%	&
%	%	\begin{pmatrix}
%		%		L_{\tau} & 0&0&0&0&0 \\
%		%			0&L_{\tau} &0&0&0&0 \\
%		%				0&0&L_{\tau}  &0&0&0 \\
%		%					0&0&0&L_{\tau} &0&0 \\
%		%						0&0&0&0&L_{\tau} &0 \\
%		%							 0&0&0&0&0&L_{\tau}  \\
%		%	\end{pmatrix}
%	%	\begin{pmatrix}
%		%		\delta_{\tau} E_x^{n+1} \\
%		%			\delta_{\tau} E_y^{n+1} \\
%		%				\delta_{\tau} E_z^{n+1} \\
%		%		\delta_{\tau} H_x^{n+3/2} \\
%		%			\delta_{\tau} H_y^{n+3/2}\\
%		%				\delta_{\tau} H_z^{n+3/2}
%		%	\end{pmatrix}=
%	%\frac{1}{h}
%	%AD\phi
%	%\end{align*}
%	where 
%	$$P=
%	\begin{pmatrix}
%		0 & 1&1 \\
%		1&0 &1 \\
%		1&1&0 
%	\end{pmatrix}\otimes A^{-1}=
%	\begin{pmatrix}
%		0 & A^{-1}&A^{-1} \\
%		A^{-1}&0 &A^{-1} \\
%		A^{-1}&A^{-1}&0 
%	\end{pmatrix},\quad
%	D=
%	\begin{pmatrix}
%		0 & -D_z&D_y \\
%		D_z&0 &-D_x \\
%		-D_y&D_x&0  \\
%	\end{pmatrix}.
%	$$
%	The discretization for $$\Delta \phi -\frac{24}{\tau^2}\phi=\frac{24}{\Delta \tau ^2}F$$ is
%	given by the three-dimensional analog of \cite[(11)]{singer_turkel}
%	$$
%	\frac{D_{xx}\phi}{1+\frac{h^2}{12}D_{xx}}+\frac{D_{yy}\phi}{1+\frac{h^2}{12}D_{yy}}
%	+\frac{D_{zz}\phi}{1+\frac{h^2}{12}D_{zz}}-\frac{24}{\Delta \tau^2}\phi
%	=\frac{24}{\Delta \tau ^2}F.
%	$$
%	We have the following:
%	\begin{enumerate}
%		\item  $|\lambda_{\max}(A^{-1})| \frac{24}{20}$.
%		\item  $|\lambda_{\max}(D)|\leq 2\sqrt{2}$
%	\end{enumerate}
%	
%	
%	The matrices $D_x,D_y,D_z$ are toplitz matrices with two dominant diagonals $1$ and $-1$, and $A^{-1}$ is the inverse of the Pad\'e operator for first derivative in Appendix \ref{appendix}.
%	
	%Then,
	%the discretization for $$\Delta \phi -\frac{24}{\tau^2}\phi=\frac{24}{\Delta \tau ^2}F$$ is
	%given by the three-dimensional analog of \cite[(11)]{singer_turkel}
	%$$
	%\frac{D_{xx}\phi}{1+\frac{h^2}{12}D_{xx}}+\frac{D_{yy}\phi}{1+\frac{h^2}{12}D_{yy}}
	%+\frac{D_{zz}\phi}{1+\frac{h^2}{12}D_{zz}}-\frac{24}{\Delta \tau^2}\phi
	%=\frac{24}{\Delta \tau ^2}F.
	%$$
	
	
	
	
	\section{Example: Transverse magnetic waves  in $[0,1]^3$}\label{sec:TE}
	Assuming that $E_x=E_y=H_z=\vec{J}=0$ we can reduce Maxwell equations to a two-dimensional problem for $E_z(t,x,y), H_x(t,x,y), H_y(t,x,y)$ where  $(x,y)\in[0,1]^2$, and
	$x_0=y_0=0$, and $x_{N-1}=y_{N-1}=0$.
	The rescaled equations are therefore 
	
	\begin{enumerate}
		\item  $ \frac{\partial E_z}{\partial \tau }=Z\left (
		\frac{\partial H_y}{\partial x}-\frac{\partial H_x}{\partial y}
		\right)$
		\item 	  $ \frac{\partial H_x}{\partial \tau }=-\frac{1}{Z}\frac{\partial E_z}{\partial y}$
		\item 	  $ \frac{\partial H_y}{\partial \tau }=\frac{1}{Z}\frac{\partial E_z}{\partial x}$,
		
	\end{enumerate}
	and the scheme  reads as the following consecutive steps.
	\begin{enumerate}
		\item  $ E_z^{n+1}=E_z^n(i,j)+Z\Delta \tau \cdot L^{-1}_{\tau}\left (\frac{\partial H_y}{\partial x}-
		\frac{\partial H_x}{\partial y}
		\right)^{n+\frac{1}{2}}$\\[1mm]
		\item $ H_x^{n+3/2}=H_x^{n+\frac{1}{2}}+\frac{\Delta \tau}{Z} \cdot  L^{-1}_{\tau}\left (-
		\frac{\partial E_z}{\partial y}
		\right)^{n+1} $\\[1mm]
		\item $ H_y^{n+3/2}=H_y^{n+\frac{1}{2}}+\frac{\Delta \tau}{Z} \cdot  L_{\tau}^{-1}\left (
		\frac{\partial E_z}{\partial x}
		\right)^{n+1} $.
	\end{enumerate}
	\begin{figure}[!t]
		\centering
		\includegraphics[scale=0.7]{figures/Ez.eps}
		\caption{
			The numerical domain for step 1.
			Squares denote boundary points. x denotes interior points of the numerical domain.  The points where $H_x$ and $H_y$ are evaluated (little blue and red dots resp.) are shown for reference.}
		\label{fig:E}
		
	\end{figure}
	
	As mentioned before, these  elliptic equations  can be solved  to fourth order   if the boundary conditions of the solutions and of the non-homogeneous term are known to fourth order \cite{singer_turkel}.
	Let us examine each step in the scheme separately:\\[1mm]
	{\bf Step 1 } (See Figure \ref{fig:E}):  
	The numerical domain: $\{(x_i,y_j)\}_{i,j\in [N]}$; boundary conditions :  the boundary values of $\delta_{\tau} E_z$ are given explicitly (=0) from the boundary conditions $\vec{\nu}\times \vec{E}=0$; non-homogeneous term:
	$\left (\frac{\partial H_y}{\partial x}-\frac{\partial H_x}{\partial y}\right )^{n+\frac{1}{2}}$ can be estimated using Pad\'e approximation (see  Appendix \ref{appendix}) at all points 
	$(x_i,y_j)$
	which do not belong to $\partial [0,1]^2$.  However, at these boundary points, the equations readily imply that  
	$\left (\frac{\partial H_y}{\partial x}-\frac{\partial H_x}{\partial y}\right )^{n+\frac{1}{2}}=0$ on $\partial [0,1]^2$.\\[1mm]
	{\bf Step 2} (See Figure \ref{fig:Hx}): 
	The numerical domain: $\{(x_i,y_{j+\frac{1}{2}})\}_{i\in [N],j\in [N-1]}$; 
	non-homogeneous term:  
	$\left (\frac{E_z}{\partial y}\right)^{n+1} $
	can be approximated  using Pad\'e approximation  in Appendix \ref{appendix} at all discretization points;
	boundary conditions:
	we  estimate 
	$\frac{H_x^{n+3/2}-H_x^{n+\frac{1}{2}}}{\Delta \tau}$ on the boundary as follows.
	The equation  
	$$
	\frac{\partial H_x}{\partial \tau}=-\frac{1}{Z}\frac{\partial E_z}{\partial y}
	$$
	implies that $H_x(0,y)$ and $H_x(1,y)$ do not depend on $\tau$, that is 
	$$\frac{H_x^{n+3/2}-H_x^{n+\frac{1}{2}}}{\Delta \tau}=0$$ if $x=0$ or $x=1$.
	On the remaining part of the boundary ($y=0,1$)  we use Taylor series:
	$$
	\frac{	H_x^{n+3/2}-H_x^{n+\frac{1}{2}}}{\Delta \tau}=
	\left(\frac{\partial H_x}{\partial \tau}\right)^{n+1}+
	\frac{\Delta \tau^2}{24}\left(\frac{\partial^3 H_x}{\partial \tau^3}\right)^{n+1}+O(\Delta \tau^4).
	$$
	%$$
	%=\left(\frac{\partial H_x}{\partial t}\right)^{n+1}\left(1+
	%\frac{\Delta t^2}{24}\cdot 
	%\left (\frac{\partial^2 H_x}{\partial t^2}\right)^{n+1}+
	%O(\Delta t^4)
	%\right)
	%$$
	Recall that
	$\left(\frac{\partial H_x}{\partial \tau}\right)^{n+1}=-\frac{1}{Z}\left(\frac{\partial E_z}{\partial y}\right)^{n+1}$ has been already approximated to fourth order. 
	If one uses just the approximation 
	$$
	\frac{	H_x^{n+3/2}-H_x^{n+\frac{1}{2}}}{\Delta \tau}=
	\left(\frac{\partial H_x}{\partial \tau}\right)^{n+1}
	$$
	then the boundary conditions are of order $h^4+\Delta \tau^2$.
	
	The remaining term 
	can be  approximated to order $\Delta \tau^2 h^2$ as follows.
	The equations imply that   $\frac{\partial^2E_z}{\partial \tau^2}=\Delta E_z$, and therefore
	$$
	\frac{\partial^3 H_x}{\partial \tau^3}=
	-\frac{1}{Z}\Delta \frac{\partial E_z }{\partial y}.
	$$
	The Laplacian of $\frac{\partial E_z }{\partial y}$ can be estimated at the boundary points  to order $h^2$ using central and one-sided schemes for second-order derivatives.
	In such a case the boundary conditions are of order $h^4+\Delta \tau^2h^2.$
	The numerical simulations   presented in  Table \ref{table:conv_rates} show the affect of the boundary conditions accuracy on the convergence rates.\\[1mm]
	{\bf Step 3} : \\The numerical domain including the boundary: $\{(x_{i+\frac{1}{2}},y_{j})\}_{i\in [N-1],j\in [N]}$; the boundary terms and the boundary conditions are treated by a similar way to step 2.
	
	\begin{figure}[!t]
		\centering
		\begin{subfigure}[h]{0.4\textwidth}
			\includegraphics[scale=0.45]{figures/Hx.eps}
		\end{subfigure}
		\begin{subfigure}[h]{0.4\textwidth}
			\includegraphics[scale=0.45]{figures/Hy.eps}
		\end{subfigure}
		\caption{The numerical domains for steps 2,3.
			Squares denote boundary points of the numerical domain. x denotes interior points. The points $(x_i,y_j)$ (little black dots) where we evaluate $E_z$ are shown for reference.}
		\label{fig:Hx}
		
	\end{figure}
	\subsection{Numerical simulations}
	We test analytical solutions in the case $Z=1$,  
	$(k_x,k_y)=(4,3)$ and $\omega:=\sqrt{k_x^2+k_y^2}$.
	Namely, 
	\begin{align*}
		&
		E_z=\cos(\omega \tau)
		\sin{\left( \pi  k_x x\right) } \sin{\left( \pi  k_y y\right) }
		\\ &
		H_x=-\frac{\sin( \omega \tau)}{\omega}
		\pi k_y \sin( \pi  k_x x)
		\cos{\left( \pi  k_y y\right) } \\&
		H_y=\frac{\sin\left( \omega \tau\right) }{\omega} \pi  k_x \cos{\left( \pi  k_x x\right) } \sin{\left( \pi  k_y y\right) }. \\&
	\end{align*}
	
	
	We use the mean absolute error 
	$$
	\frac{1}{N_{\tau}\cdot N^2}\sum_{i,j,n=0}^{N,N,N_{\tau}}\left | F^n_{\mathrm{numer.}}(i,j)-F_{\mathrm{true}}^n(i,j) \right|
	$$
	$(\vec{F}=(E_z,H_x,H_y))$ to calculate the (log-)convergence rates of our schemes.
	%	\begin{table} [h!]
		%	\centering
		%	\begin{tabular}{|p{0.7cm}|p{0.7cm}|p{2cm}|p{2cm}| p{2cm}|}
			%		\hline
			%$h$ & $\mathrm{CFL}$ &$\Delta \tau^2+h^4$&$\Delta \tau^2h^2+h^4$& analytic
			%		\\ [0.5ex] 
			%		\hline 
			%$\frac{1}{20}$ & 0.5& & & \\
			%$\frac{1}{25}$ & 0.5 &3.74 &5.51&5.27\\
			%$\frac{1}{30}$ & 0.5 & 3.50 &5.55& 5.38 \\
			%$\frac{1}{35}$ & 0.5 &3.38 & 5.57&5.30\\
			%$\frac{1}{40}$ & 0.5& 3.31 &5.55 &5.22\\
			%\hline 
			%	\end{tabular}
		%	\caption{ Converge rates with several types of boundary conditions for $\delta_\tau \vec{H}$ in steps 2 and 3:	 
			%	Here $\Delta \tau=\frac{h}{2}$ where $h=\frac{1}{20},\frac{1}{25},\frac{1}{30},
			%	\frac{1}{35}, \frac{1}{40}
			%	$
			%	and final time $T=1$.}
		%	\label{table:conv_rates}
		
		
		%\end{table}
		\begin{table} [h!]
			\centering
			\begin{tabular}{|p{0.7cm}|p{0.7cm}|p{2cm}|p{2cm}|}
				\hline
				$h$ & $\mathrm{CFL}$ &$\Delta \tau^2+h^4$&$\Delta \tau^2h^2+h^4$
				\\ [0.5ex] 
				\hline 
				$\frac{1}{16}$ & 0.5& &  \\[0.5mm]
				$\frac{1}{32}$ & 0.5 &3.70 &5.45\\[0.5mm]
				$\frac{1}{64}$ & 0.5 & 3.21 &5.23 \\[0.5mm]
				$\frac{1}{128}$ & 0.5 &3.03 & 3.98\\[0.5mm]
				$\frac{1}{256}$ & 0.5& 2.99 &3.93 \\[0.5mm]
				\hline 
			\end{tabular}
			\caption{ Converge rates with several types of boundary conditions for $\delta_\tau \vec{H}$ in steps 2 and 3:	 
				Here $\Delta \tau=\frac{h}{2}$ where $h=\frac{1}{16},\frac{1}{32},\frac{1}{64},
				\frac{1}{128}, \frac{1}{256}
				$
				and final time $T=1$.}
			\label{table:conv_rates}
			
			
		\end{table}
		\begin{table} [h!]
			\centering
			\begin{tabular}{|p{0.7cm}|p{0.7cm}|p{2cm}| |p{4cm}|}
				\hline
				$h$ & $\mathrm{CFL}$ &error& convergence rate
				\\ [0.5ex] 
				\hline 
				$\frac{1}{16}$ & 0.5& 2.9e-3&  \\[0.5mm]
				$\frac{1}{32}$ & 0.5 &6.7e-5 &5.45\\[0.5mm]
				$\frac{1}{64}$ & 0.5 & 1.8e-6 &5.23 \\[0.5mm]
				$\frac{1}{128}$ & 0.5 &1.1e-7 & 3.98\\[0.5mm]
				$\frac{1}{256}$ & 0.5& 7.5e-9 &3.93 \\[0.5mm]
				\hline \hline
				$\frac{1}{16}$ & 0.5& 0.38&  \\[0.5mm]
				$\frac{1}{32}$ & 0.5 &0.22 &0.75\\[0.5mm]
				$\frac{1}{64}$ & 0.5 & 5.2e-3 &5.4 \\[0.5mm]
				$\frac{1}{128}$ & 0.5 &2.2e-4 & 4.52\\[0.5mm]
				$\frac{1}{256}$ & 0.5& 1.6e-5 &3.822 \\[0.5mm]
				\hline
			\end{tabular}
			\caption{ Converge rates with $k_x=4,k_y=3$ (upper table)
				and  $k_x=17,k_y=18$.
				Here $\Delta \tau=\frac{h}{2}$ where $h=\frac{1}{16},\frac{1}{32},\frac{1}{64},
				\frac{1}{128}, \frac{1}{256}
				$
				and final time $T=1$.}
			\label{table:conv_rates_2}
			
			
		\end{table}
		\subsection{Comparison with non-compact fourth order scheme and with data-driven scheme (AI(2,3)) trained on high wave-numbers}
		\begin{table} [h!]
			\centering
			\begin{tabular}{|p{0.7cm}|p{0.7cm}|p{4cm}|p{4cm}|p{4cm}|}
				\hline
				$h$ & $\mathrm{CFL}$ &compact-4th& AI(2,3)&non-compact-4th
				\\ [0.5ex] 
				\hline 
				$\frac{1}{16}$ & 0.5& 6.9e-5&  6e-5&8.7e-5 \\[0.5mm]
				$\frac{1}{32}$ & 0.5 &1.5e-6 &6.9e-6&1.18e-5 \\[0.5mm]
				$\frac{1}{64}$ & 0.5 & 6.8e-8 &8.2e-7& 1.5e-6 \\[0.5mm]
				$\frac{1}{128}$ & 0.5 &4.9e-9 & 1e-7&1.9e-7 \\[0.5mm]
				$\frac{1}{256}$ & 0.5& 3.7e-10 &1.2e-8&2.4e-8 \\[0.5mm]
				\hline \hline
				$\frac{1}{16}$ & 0.5& 2.9e-3&  2.6e-4&8.1e-4 \\[0.5mm]
				$\frac{1}{32}$ & 0.5 &6.7e-5 &2.2e-5&1.1e-4 \\[0.5mm]
				$\frac{1}{64}$ & 0.5 & 1.8e-6 &1.8e-6& 1.4e-5 \\[0.5mm]
				$\frac{1}{128}$ & 0.5 &1.1e-7 & 1.8e-7&1.8e-6 \\[0.5mm]
				$\frac{1}{256}$ & 0.5& 7.5e-9 &1.9e-8&2.2e-7 \\[0.5mm]
				\hline \hline
				$\frac{1}{16}$ & 0.5& 0.38& 3.4e-3 &4.4e-3\\[0.5mm]
				$\frac{1}{32}$ & 0.5 &0.22 &2.9e-3&3.4e-3 \\[0.5mm]
				$\frac{1}{64}$ & 0.5 & 5.2e-3 &1e-4& 1.7e-3\\[0.5mm]
				$\frac{1}{128}$ & 0.5 &2.2e-4 & 2.84e-6&2.8e-4\\[0.5mm]
				$\frac{1}{256}$ & 0.5& 1.6e-5 &1.02e-6& 3.5e-5\\[0.5mm]
				\hline
			\end{tabular}
			\caption{ Errors for solutions  with wave-numbers $k_x=1,k_y=2$ (upper table), $k_x=4,k_y=3$ (middle table),
				and  $k_x=17,k_y=18$.
				Here $\Delta \tau=\frac{h}{2}$ where $h=\frac{1}{16},\frac{1}{32},\frac{1}{64},
				\frac{1}{128}, \frac{1}{256}
				$
				and final time $T=1$.}
			\label{table:conv_rates_3}
		\end{table}
		\newpage
		%\section{Convergence rates with boundary conditions for $H$ of order 
			%	$dt^2+h^4$}
		%We test analytical solutions in the case $c=1$,  such that 
		%$(k_x,k_y)=(4,3)$ and $\Omega=\sqrt{k_x^2+k_y^2}$.
		%We take $dt=\frac{h}{2}$ where $h=\frac{1}{20},\frac{1}{25},\frac{1}{30},
		%\frac{1}{35}, \frac{1}{40}$ from time 0 to time 1.
		%We consider the mean absolute error, that is 
		%$$
		%\frac{1}{N_t\cdot N^2}\sum_{i,j,n=0}^{N,N,N_t}\left | F^n(i,j)-F_{\mathrm{true}}^n(i,j) \right|
		%$$
		%$(\vec{F}=(E_z,H_x,H_y))$
		%to calculate the convergence rates as a function of $1/h$:
		%$$
		%(3.02, 3.15, 3.14,  3.15)
		%$$
		
		%	\begin{table} [h!]
			%	\centering
			%	\begin{tabular}{|p{0.7cm}|p{0.7cm}|p{0.7cm}| p{5cm}|p{5cm}|}
				%		\hline
				%		$N$& $N_t$ & $T$ &b.c for $H$: $O(dt^2+h^2)$& 
				%		b.c for $H$: $O(dt^2+h^4)$\\
				%		
				%		
				%
				%		
				%		
				%		
				%		\hline
				%	\end{tabular}
			%	\caption{Convergence rates
				%	}
			%	\label{table:tab3}
			%\end{table}
			\appendix
			\section{}\label{appendix}
			Assume that $f(x)$ is known at N points $x_0,x_1, ..,x_{N-1}$
			and we want to estimate to fourth order $f'(x)$ at N-1 points
			$x_{\frac{1}{2}},..,x_{N-3/2}$. This can be done  by solving the system of equations
			\begin{equation}\label{eq:lhs}
				\frac{1}{24}
				\begin{pmatrix}
					26     & -5     & 4      & -1     & 0      & \dots  & 0      \\
					1      & 22     & 1      & 0      & 0      & \dots  & 0      \\
					0      & 1      & 22     & 1      & 0      & \dots  & 0      \\
					\vdots & \ddots & \ddots & \ddots & \ddots & \ddots & \vdots \\
					0      & 0      & \dots  & 1      & 22     & 1      & 0      \\
					0      & 0      & \dots  & 0      & 1      & 22     & 1      \\
					0      & 0      & \dots  & -1     & 4      & -5     & 26
				\end{pmatrix}
				\begin{pmatrix}
					f'(x_{\frac{1}{2}}) \\
					f'(x_{3/2}) \\
					\cdots \\
					\cdots\\
					\cdots \\
					\cdots \\
					f'(x_{N-3/2})
				\end{pmatrix}=
				\frac{1}{\Delta x}
				\begin{pmatrix}
					f(x_{1})-f(x_0)  \\
					\cdots \\
					\cdots\\
					\cdots \\
					\cdots \\
					\cdots \\
					f(x_{N-1)}-f(x_{N-2})
				\end{pmatrix}.
			\end{equation}
	
			%By Taylor approximation:
			%$$
			%\frac{f(x+h/2)-f(x-h/2)}{h}=f'(x)+\frac{f'''(x)h^2}{24}+O(h^4)
			%$$
			%$$=
			%f'(x)+\frac{1}{24}(f'(x+h)-2f'(x+h)+f'(x-h))+O(h^4)
			%$$
			\section{}\label{appendxib}
				We first achieve fourth order accuracy in time using a Taylor expansion:
			\begin{align*}
				\left (\frac{\partial D_z}{\partial t} \right)^{n+\frac{1}{2}} &= \frac{D_z^{n+1} - D_z^{n}}{\Delta t} - \frac{(\Delta t^2)}{24} \frac{\partial D^3_z}{\partial t^3} + O(\Delta t)^4 \\
				&= \frac{D_z^{n+1} - D_z^{n}}{\Delta t} - \frac{(\Delta t^2)}{24} \partial_{tt} \left(\frac{\partial H_y}{\partial x}
				- \frac{\partial H_x}{\partial y} -  J_z \right)
				+ O(\Delta t)^4
			\end{align*}
			We now use the relation $H \!=\! \frac{1}{\mu} B$. For simplicity we define
			\begin{equation}
				\label{Dhat}
				\delta_t D_z: = \frac{D_z^{n+1} - D_z^{n}}{\Delta t} .
			\end{equation}
			We then get
			\begin{align*}
				\left (\frac{\partial D_z}{\partial t} \right)^{n+\frac{1}{2}}  &= \delta_t D_z - \frac{\Delta t^2}{24} \partial_{tt} \left(\frac{\partial {B_y \over \mu}}{\partial x }
				- \frac{\partial{B_x \over \mu}}{\partial y} - J_z \right) + O(\Delta t)^4 \\
				&= \delta_t D_z - \frac{\Delta t^2}{24} \frac{\partial}{\partial t} \left(\frac{\partial^2 {B_y \over \mu}}{\partial x \partial t}
				- \frac{\partial^2 {B_x \over \mu}}{\partial y \partial t} - \frac{\partial J_z}{\partial t} \right) + O(\Delta t)^4.
			\end{align*}
			Recall that 
			
			$$
			\frac{\partial B_y}{\partial t}=\frac{\partial E_z}{\partial x}-\frac{\partial E_x}{\partial z},\qquad
			\frac{\partial B_x}{\partial t}=\frac{\partial E_y}{\partial z}-\frac{\partial E_z}{\partial y}
			$$
			and since $\mu$ does not depend on $t$,
			$$
			\frac{\partial (B_y/\mu)}{\partial t}=
			\frac{1}{\mu}\left (
			\frac{\partial E_z}{\partial x}-\frac{\partial E_x}{\partial z}
			\right), \qquad
			\frac{\partial (B_x/\mu)}{\partial t}=
			\frac{1}{\mu}\left (
			\frac{\partial E_y}{\partial z}-\frac{\partial E_z}{\partial y}
			\right) .
			$$
			
			Hence
			$$
			\left(\frac{\partial^2 {B_y \over \mu}}{\partial x \partial t}
			- \frac{\partial^2 {B_x \over \mu}}{\partial y \partial t}  \right)=
			\left (
			\frac{\partial}{\partial x} \frac{1}{\mu}\left (
			\frac{\partial E_z}{\partial x}-\frac{\partial E_x}{\partial z}
			\right)-\frac{\partial }{\partial y} \frac{1}{\mu}\left (
			\frac{\partial E_y}{\partial z}-\frac{\partial E_z}{\partial y}
			\right) 
			\right)
			$$
			\begin{equation}\label{eq:mu_diff}
				=\frac{1}{\mu}
				\left(
				\frac{\partial^2 E_z}{\partial x^2}
				+\frac{\partial^2 E_z}{\partial y^2}
				-\frac{\partial^2 E_x}{\partial x\partial z}
				-\frac{\partial^2 E_y}{\partial y\partial z}
				\right)+
				\frac{\partial (1/\mu)}{\partial x}\left ( \frac{\partial E_z}{\partial x}-\frac{\partial E_x}{\partial z}\right)
				-
				\frac{\partial (1/\mu)}{\partial y}\left ( \frac{\partial E_y}{\partial z}-\frac{\partial E_z}{\partial y}\right).
			\end{equation}
			After adding and subtracting $\frac{\partial^2 E_z  }{\partial z^2}$
			in the first parenthesis in \eqref{eq:mu_diff}, and adding and subtracting
			$\frac{\partial (1/\mu)}{\partial z}\frac{\partial E_z}{\partial z}$
			we obtain that  \eqref{eq:mu_diff} equals to
			$$
			div(\mu^{-1}\nabla E_z)-div(\mu^{-1}\frac{\partial \vec{E}}{\partial z}).
			$$
			This yields
			\begin{align*}
				&
				\left ( \frac{\partial H_y}{\partial x}-
				\frac{\partial H_x}{\partial y}-J_z
				\right)^{n+\frac{1}{2}}=\\&
				\frac{\partial D_z^{n+\frac{1}{2}}}{\partial t} =
				\delta_t D_z - \frac{\Delta t^2}{24} \frac{\partial}{\partial t}
				\left(
				div(\mu^{-1}\nabla E_z)-div(\mu^{-1}\frac{\partial \vec{E}}{\partial z})
				-\frac{\partial J_z}{\partial t}
				\right)+O(\Delta t	)^4= \\&
				\varepsilon \delta_t  E_z - \frac{\Delta t^2}{24}
				\left(
				div(\mu^{-1} \nabla \delta_t  E_z)
				- div\left (\mu^{-1}\frac{\partial ^2\vec{E}}{\partial t \partial z}
				\right)
				-\frac{\partial^2 J_z}{\partial t^2}
				\right)
				+O(\Delta t	)^4.
			\end{align*}
			The term 
			$$
			\left ( \frac{\partial H_y}{\partial x}-
			\frac{\partial H_x}{\partial y}-J_z
			\right)^{n+\frac{1}{2}}
			$$
			can be approximated to fourth order using the Pade approximation given in Appendix \ref{appendix} \cite{yefet_turkel}(see Section \ref{sec:TE}). 
			The term
			$$
			div\left(\mu^{-1}\frac{\partial^2\vec{E}}{\partial t \partial z}\right )^{n+\frac{1}{2}}
			=div\left(
			\mu^{-1}\frac{\partial }{ \partial z}\frac{1}{\varepsilon}(
			\nabla \times \vec{H} -\vec{J}
			)
			\right)^{n+\frac{1}{2}}
			$$
			can be estimated from the previous
			time  step to the second order  accuracy in $h$ since it is already multiplied by $\Delta t ^2$. 
			In fact, since $div(\nabla\times H)=0$ we have to approximate to second order only first and second derivatives  of $\vec{H}$.
			Hence,
			$E_z^{n+1}=E^n+\Delta t\cdot \delta_t E_z$ where 
			$\delta_t E_z$ solves the non-negative symmetric  elliptic equation
			\begin{equation}\label{eq:elliptic_div}
				\varepsilon \delta_t E_z-\frac{\Delta t^2}{24}div(\mu^{-1}\nabla \delta_t E_z)=F^{n+\frac{1}{2}},
			\end{equation}
			where 
			$$
			F^{n+\frac{1}{2}}=
			\left ( \frac{\partial H_y}{\partial x}-
			\frac{\partial H_x}{\partial y}-J_z
			\right)^{n+\frac{1}{2}}+\frac{\Delta t^2}{24}\left (
			div\left(
			\mu^{-1}\frac{\partial }{ \partial z}\frac{1}{\varepsilon}(
			\nabla \times \vec{H} -\vec{J}
			)
			\right )-\frac{\partial^2J_z}{\partial t^2}
			\right)^{n+\frac{1}{2}}.
			$$
			For fourth order compact schemes for \eqref{eq:elliptic_div} see \cite{britt_tsy_tur}. 
				\subsection{Boundary conditions for $\delta_t \vec{E}$ and  $\delta_t \vec{H}$}
			In order to solve \eqref{eq:elliptic_div} to fourth order  one has to supply certain (Dirichlet) boundary conditions for $\delta_t \vec{E}$ and  $\delta_t \vec{H}$ to fourth order.
			We will show the construction in the first  time step. The next steps  will then follow by the staggered structure.
			We assume that $E^0$ and $H^{\frac{1}{2}}$ are given to fourth order.
			For $H^{\frac{1}{2}}$ this can be done using  standard Taylor approximation.
			Next,
			$$
			\frac{\vec{E}^{1}-\vec{E}^{0}}{\Delta t}=\frac{\partial \vec{E}^{\frac{1}{2}}}{\partial t}+
			\Delta t^2\frac{\partial^3 \vec{E}^{\frac{1}{2}}}{\partial t^3}+O(\Delta t^4).
			$$
			The equation for $\vec{E}$ imply that 
			$$
			\frac{\partial \vec{E}^{\frac{1}{2}}}{\partial t}=\varepsilon^{-1}\cdot \nabla \times \vec{H}^{\frac{1}{2}}.
			$$
			The latter term is known to fourth order  and therefore can be approximated using Pad'e scheme in Appendix \ref{appendix} to fourth order. 
			Next we consider the term 
			\begin{align*}
				&
				\frac{\partial^3 \vec{E}^{\frac{1}{2}}}{\partial t^3} =
				\varepsilon^{-1}\nabla \times  \frac{\partial^2 H^{\frac{1}{2}}}{\partial t^2}=
				\\&
				-\varepsilon^{-1}\nabla\times \mu^{-1}
				\nabla \times \frac{\partial \vec{E}^{\frac{1}{2}}}{\partial t}=
				-\varepsilon^{-1}\nabla\times \mu^{-1} \nabla \times
				\varepsilon^{-1}\nabla\times \vec{H}^{\frac{1}{2}}.
			\end{align*}
			The term $ \nabla\times \vec{H}^{\frac{1}{2}}$ is known to fourth order and therefore it is sufficient to approximate its  second-derivatives to second order. Since 
			$\frac{\partial^3 \vec{E}^{\frac{1}{2}}}{\partial t^3}$ is multiplied by $\Delta t^2$ we have accomplished our estimation to order $h^4+\Delta t^2h^2$.
			We demonstrate the above construction   in the following section.
			
			
			\begin{center}
				{\bf Acknowledgments}
			\end{center}
			\begin{thebibliography}{99}
				
				
				\bibitem{britt_tsy_tur}					
				Britt, Steven, Semyon Tsynkov, and Eli Turkel. "Numerical simulation of time-harmonic waves in inhomogeneous media using compact high order schemes." Communications in Computational Physics 9.3 (2011): 520-541.					
				\bibitem{Morton}
				Morton,K.W.,Mayers,D.F.:NumericalSolutionofPartialDifferentialEquations,2ndedn.Cambridge
				University Press, (2005)
				\bibitem{chen}
				Chen, Wenbin, Xingjie Li, and Dong Liang. "Energy-conserved splitting FDTD methods for Maxwell?s equations." Numerische Mathematik 108.3 (2008): 445-485.
				
				\bibitem{rolf}
				Leis, Rolf. Initial boundary value problems in mathematical physics. Courier Corporation, 2013.
				\bibitem{sakka}
				Sakkaplangkul, Puttha, and V. Bokil. "CONVERGENCE ANALYSIS OF YEE-FDTD SCHEMES FOR 3D MAXWELL?S EQUATIONS IN LINEAR DISPERSIVE MEDIA." International journal of numerical analysis and modeling 18.4 (2021).	
				\bibitem{singer_turkel}
				I. ~Singer, and E.~ Turkel. "High-order finite difference methods for the Helmholtz equation." Computer methods in applied mechanics and engineering 163.1-4 (1998): 343-358.					
				\bibitem{yee}
				K.~Yee, Numerical solution of initial boundary value problems involving Maxwell's equations in isotropic media, {\em IEEE Transactions on antennas and propagation} {\bf 14.3} (1966), 302-307.
				\bibitem{yefet_turkel}
				A.~Yefet and E.~Turkel,
				Fourth Order Compact Implicit Method for the Maxwell Equations with Discontinuous Coefficients
				{\em Applied Numerical Mathematics}  {\bf 33} (2000), 125--134.
			\end{thebibliography}
		\end{document}