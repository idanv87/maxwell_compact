\documentclass[12pt,reqno]{amsart}

\usepackage{graphicx}
\graphicspath{ {./images/} }

\usepackage{float}
\usepackage{amsfonts,amssymb,amsmath,amsopn,amsthm,graphicx}
\usepackage{amsxtra, mathrsfs}
\usepackage{colordvi}
\usepackage[usenames,dvipsnames]{color}
\usepackage{amsfonts,amssymb,amsbsy,amsmath,amsthm,dsfont}
%%%%%%%%%%%
\usepackage{xspace}
\usepackage{pgfplots}
\usepackage{lipsum}
\usepackage[many]{tcolorbox}
\usetikzlibrary{decorations.pathreplacing}




\linespread{1.1} \numberwithin{equation}{section}


\newcommand\red[1]{\textcolor{red}{#1}}
\newcommand{\Gluck}{Gl\H{u}ck\xspace}
\newcommand{\ra}{def}

\newtcolorbox{leftbrace}{%
	enhanced jigsaw, 
	breakable, % allow page breaks
	frame hidden, % hide the default frame
	overlay={%
		\draw [
		decoration={brace,amplitude=0.5em},
		decorate,
		ultra thick,
		]
		% right line
		(frame.south west)--(frame.north west);
	},
	% paragraph skips obeyed within tcolorbox
	parbox=false,
}


%%%%%%%%%
\usepackage{amssymb,amsmath,graphicx,amsthm,a4wide,wrapfig,caption,subcaption,epstopdf}
%\usepackage{amsmath,amssymb,amsfonts,amscd,hyperref,color}
\usepackage[latin1]{inputenc}
%\usepackage[mtoscr]{mtpro2}
\usepackage{enumitem}

%\usepackage[abs]{overpic}
%\usepackage{graphicx}
%\usepackage{verbatim}
%\usepackage{amscd}
\usepackage{verbatim}
%%%%%%%%%%%%%%%%%%%%%%%%%%
%%%%labels%%%%%%%%
%\usepackage[notref,notcite]{showkeys}
%%%%%%%%%%%%%%%




\DeclareRobustCommand{\rchi}{{\mathpalette\irchi\relax}}
\newcommand{\irchi}[2]{\raisebox{\depth}{$#1\chi$}} % inner command, used by \rch
\usepackage{lineno}
\usepackage{textcomp}
\usepackage{mathtools,hyperref}
\usepackage{cleveref}
%\hyperref[label_name]{''link text''}

%\usepackage[utf8]{inputenc}
%\usepackage[english]{babel}
\usepackage{setspace,esint}
%\usepackage{graphicx}
%\usepackage{mathtools}
%\usepackage{caption}
%\usepackage{subcaption}
%\graphicspath{{photos/}}
%\usepackage[usenames, dvipsnames]{color}
%\usepackage{}
%\usepackage{enumitem}
%\newcommand\mysymbol[3]{\protected\gdef#1{#2}%	\item[$#2$]#3}










%%%%%%%%%%%%%%%

%\linespread{1.1} \numberwithin{equation}{section}
%\setlength{\voffset}{-.7truein}
%\setlength{\textheight}{8.8truein}
%\setlength{\textwidth}{6.1truein}
%\setlength{\hoffset}{-.7truein}


%%%%%
\newcommand\blue[1]{\textcolor{blue}{#1}}
%%%%%%%%%%%%%%%%%

\newcommand{\Hmm}[1]{\leavevmode{\marginpar{\tiny%
			$\hbox to 0mm{\hspace*{-0.5mm}$\leftarrow$\hss}%
			\vcenter{\vrule depth 0.1mm height 0.1mm width \the\marginparwidth}%
			\hbox to
			0mm{\hss$\rightarrow$\hspace*{-0.5mm}}$\\\relax\raggedright #1}}}
%%%%%%%%%%%%%%%%%%%%
\newcommand{\wc}{{\overline{\Omega}}}
\newcommand{\bb}{{\mathbf{\bar {b}}}}
\newcommand{\bt}{{\mathbf{\tilde{b}}}}
\newcommand{\Gwb}{{\overline{\Omega}}}
\newcommand{\Hd}{\,H^1_{\Pw_{\mathrm{Dir}}}(\Gw)}

%%%%%%%%%%%%%%%%%%%%%%%
\newcommand{\im}{\mbox{\upshape Im\ }}
\newcommand{\re}{\mbox{\upshape Re\ }}
\newcommand{\tr}{\mbox{\upshape Tr }}
\newcommand{\Pw}{\partial \Omega}

\newcommand{\A}{\mathbf{A}}
\newcommand{\B}{\mathscr{B}}
\newcommand{\ball}{B}
\newcommand{\C}{\mathbb{C}}
\newcommand{\cl}{\mathrm{cl}}
\newcommand{\D}{{\rm D}}
\newcommand{\rr}{{\rm R}}
\newcommand{\Q}{\mathcal{Q}}
\newcommand{\eps}{\varepsilon}
\newcommand{\I}{\mathbb{I}}
\newcommand{\pd}{\partial}
\newcommand{\F}{\mathcal{F}}

\newcommand{\K}{\mathcal{K}}
\newcommand{\loc}{{\rm loc}}
\newcommand{\mg}{\mathrm{mag}}
\newcommand{\N}{\mathbb{N}}
\newcommand{\ope}{\mathrm{op}}
\newcommand{\R}{\mathbb{R}}
\newcommand{\sph}{\mathcal{L}}
\newcommand{\s}{\mathcal{S}}
\newcommand{\T}{\mathbb{T}}
\newcommand{\Z}{\mathbb{Z}}
\newcommand{\W}{\mathscr{W}}
\newcommand{\id}{\mathds{1}}
\newcommand{\HD}{H^{1}_{\partial \Omega_{D}}(\Gw)}
\newcommand{\curl}{{\bf curl}}
\newcommand{\acurl}{\langle a^{-1},{\bf curl_h}\rangle}
\newcommand{\e}{{\bf E}}
\newcommand{\h}{{\bf H}}
\newcommand{\J}{{\bf J}}
\newcommand{\p}{{\bf P}}
\newcommand{\pwd}{\partial\Omega_{\mathrm{Dir}}}
\newcommand{\pwr}{\partial\Omega_{\mathrm{Rob}}}
\newcommand{\wb}{\overline{\Omega}\setminus \partial\Omega_{\mathrm{Dir}} }
\newcommand{\Dir}{\mathrm{Dir}}

\newcommand{\sign}{\mathrm{sign}\,}

\newtheorem{theorem}{Theorem}[section]
\newtheorem{corollary}[theorem]{Corollary}
\newtheorem{cor}[theorem]{Corollary}
\newtheorem{thm}[theorem]{Theorem}
\newtheorem{lem}[theorem]{Lemma}
\newtheorem{lemma}[theorem]{Lemma}
\newtheorem{proposition}[theorem]{Proposition}
\newtheorem{example}[theorem]{Example}

\newtheorem{definition}[theorem]{Definition}
\newtheorem{defi}[theorem]{Definition}
\newtheorem{remark}[theorem]{Remark}

\newtheorem{rem}[theorem]{Remark}

\newtheorem{remarks}[theorem]{Remarks}


\theoremstyle{definition}

%\newtheorem{remark}[theorem]{Remark}
\newtheorem{assumption}[theorem]{Assumption}
\newtheorem{assumptions}[theorem]{Assumptions}
\newtheorem{defin}[theorem]{Definition}

%%%%%%%%%%%%%%%%%%%%%
\numberwithin{equation}{section}
\newcommand{\diver}{\mathrm{div}\,}
%\newcommand{\loc}{{\mathrm{loc}}}
\newcommand{\RN}[1]{%
	\textup{\uppercase\expandafter{\romannumeral#1}}%
}
\newcommand{\interior}[1]{\accentset{\smash{\raisebox{-0.12ex}{$\scriptstyle\circ$}}}{#1}\rule{0pt}{2.3ex}}

\newcommand{\dx}{\,\mathrm{d}x}
\newcommand{\dy}{\,\mathrm{d}y}
\newcommand{\dz}{\,\mathrm{d}z}
\newcommand{\dt}{\,\mathrm{d}t}
\newcommand{\du}{\,\mathrm{d}u}
\newcommand{\dv}{\,\mathrm{d}v}
\newcommand{\dV}{\,\mathrm{d}V}
\newcommand{\ds}{\,\mathrm{d}s}
\newcommand{\dr}{\,\mathrm{d}r}
\newcommand{\dS}{\,\mathrm{d}S}
\newcommand{\dk}{\,\mathrm{d}k}
\newcommand{\dphi}{\,\mathrm{d}\phi}
\newcommand{\dtau}{\,\mathrm{d}\tau}
\newcommand{\dxi}{\,\mathrm{d}\xi}
\newcommand{\deta}{\,\mathrm{d}\eta}
\newcommand{\dsigma}{\,\mathrm{d}\sigma}
\newcommand{\dtheta}{\,\mathrm{d}\theta}
\newcommand{\dnu}{\,\mathrm{d}\nu}
\newcommand{\dmu}{\,\mathrm{d}\mu}
\newcommand{\drho}{\,\mathrm{d}\rho}
\newcommand{\dvrho}{\,\mathrm{d}\varrho}
\newcommand{\dkappa}{\,\mathrm{d}\kappa}
\newcommand{\intr}[1]{\mathrm{int}(#1)}

%%%%%%%%%%%%%%%%%%%%%%%%
\newcommand{\core}{C_0^{\infty}(\Omega)}
\newcommand{\coredir}{C_0^{\infty}(\overline{\Omega} \setminus \pwd)}
\newcommand{\sob}{W^{1,p}(\Omega)}
\newcommand{\sobloc}{W^{1,p}_{\mathrm{loc}}(\Omega)}
\newcommand{\merhav}{{\mathcal D}^{1,p}}
\newcommand{\be}{\begin{equation}}
	\newcommand{\ee}{\end{equation}}
%\newcommand{\mysection}[1]{\section{#1}\setcounter{equation}{0}}
%%%%%%%%%%%%%%%
\newcommand{\bea}{\begin{eqnarray}}
	\newcommand{\eea}{\end{eqnarray}}
\newcommand{\bean}{\begin{eqnarray*}}
	\newcommand{\eean}{\end{eqnarray*}}
\newcommand{\thkl}{\rule[-.5mm]{.3mm}{3mm}}
%%%%%%%%%%%%%%%%%
\newcommand{\Rob}{\mathrm{Rob}}
\newcommand{\Real}{\mathbb{R}}
\newcommand{\real}{\mathbb{R}}
\newcommand{\Nat}{\mathbb{N}}
\newcommand{\ZZ}{\mathbb{Z}}
\newcommand{\Proof}{\mbox{\noindent {\bf Proof} \hspace{2mm}}}
\newcommand{\mbinom}[2]{\left (\!\!{\renewcommand{\arraystretch}{0.5}
		\mbox{$\begin{array}[c]{c}  #1\\ #2  \end{array}$}}\!\! \right )}
\newcommand{\brang}[1]{\langle #1 \rangle}
\newcommand{\vstrut}[1]{\rule{0mm}{#1mm}}
\newcommand{\rec}[1]{\frac{1}{#1}}
\newcommand{\set}[1]{\{#1\}}
\newcommand{\dist}[2]{\mbox{\rm dist}\,(#1,#2)}
\newcommand{\opname}[1]{\mbox{\rm #1}\,}
\newcommand{\supp}{\opname{supp}}
\newcommand{\mb}[1]{\;\mbox{ #1 }\;}
\newcommand{\undersym}[2]
{{\renewcommand{\arraystretch}{0.5}  \mbox{$\begin{array}[t]{c}
				#1\\ #2  \end{array}$}}}
\newlength{\wex}  \newlength{\hex}
\newcommand{\understack}[3]{%
	\settowidth{\wex}{\mbox{$#3$}} \settoheight{\hex}{\mbox{$#1$}}
	\hspace{\wex}  \raisebox{-1.2\hex}{\makebox[-\wex][c]{$#2$}}
	\makebox[\wex][c]{$#1$}   }%
%%Macros for changing font size in math.
\newcommand{\smit}[1]{\mbox{\small \it #1}}% only for letters, numbers
\newcommand{\lgit}[1]{\mbox{\large \it #1}}% only for letters, numbers
\newcommand{\scts}[1]{\scriptstyle #1}
\newcommand{\scss}[1]{\scriptscriptstyle #1}
\newcommand{\txts}[1]{\textstyle #1}
\newcommand{\dsps}[1]{\displaystyle #1}
\newcommand{\ass}[1]{Let Assumptions~\ref{assump1} hold  in a bounded Lipschitz domain $\Gw$}

%%%%%%%%%%%%%%%%%%%%%%%%%%%%%%%Macros for Greek letters.

%%%%%%%%%%%%%%%%%%%%%%%%%%%
\def\ga{\alpha}     \def\gb{\beta}       \def\gg{\gamma}
\def\gc{\chi}       \def\gd{\delta}      \def\ge{\epsilon}
\def \gth{\theta}                         \def\vge{\varepsilon}
\def\gf{\phi}       \def\vgf{\varphi}    \def\gh{\eta}
\def\gi{\iota}      \def\gk{\kappa}      \def\gl{\lambda}
\def\gm{\mu}        \def\gn{\nu}         \def\gp{\pi}
\def\vgp{\varpi}    \def\gr{\rho}        \def\vgr{\varrho}
\def\gs{\sigma}     \def\vgs{\varsigma}  \def\gt{\tau}
\def\gu{\upsilon}   \def\gv{\vartheta}   \def\gw{\omega}
\def\gx{\xi}        \def\gy{\psi}        \def\gz{\zeta}
\def\Gg{\Gamma}     \def\Gd{\Delta}      \def\Gf{\Phi}
\def\Gth{\Theta}
\def\Gl{\Lambda}    \def\Gs{\Sigma}      \def\Gp{\Pi}
\def\Gw{\Omega}     \def\Gx{\Xi}         \def\Gy{\Psi}
\def\Gwh{\Omega_h}  
%%%%%%%%%%%%%


%%%%%%%%%%%%%%%%%%%%%%%%%%%%%%%%%%%%%%%%%
\begin{document}
	
	
	\title{Fourth Order Accurate in Space and Time Compact scheme for First Order Time Dependent Maxwell Equations}
	
	\author {I. Versano}
	
	\address {School of Mathematical Sciences, Tel-Aviv University, Tel-Aviv 6997801, Israel}
	
	\email {idanversano@tauex.tau.ac.il}
	
	\author {E. Turkel}
	
	\address {School of Mathematical Sciences, Tel-Aviv University, Tel-Aviv 6997801, Israel}
	
	\email {turkel@tauex.tau,ac.il}
	
	\author {S. Tsynkov }
	
	\address {North Carolina State University, Box 8205, Raleigh, NC 27695, USA.}
	
	\email {tsynkov@math.ncsu.edu}
	%
	%\author{
		%	I. Versano {}\thanks{Corresponding author.
			%		School of Mathematical Sciences, Tel-Aviv University, Tel-Aviv 6997801, Israel, E-mail: idanv@campus.technion.ac.il} \qquad
		%	E. Turkel {}\thanks{
			%		School of Mathematical Sciences, Tel-Aviv University, Tel-Aviv 6997801, Israel, E-mail: turkel@tauex.tau,ac.il} \qquad
		%	S. Tsynkov {}\thanks{
			%		Department of Mathematics,  North Carolina State University, Box 8205, Raleigh, NC 27695, USA.
			%		tsynkov@math.ncsu.edu} 
		%}
	
	
	
	
	
	%\fi
	
	%%%%%%%%%%%%%%
	
	
	\maketitle
	\begin{abstract}
		
		
		\medskip
		
		\noindent  2000  \! {\em Mathematics  Subject  Classification.}
		.\\[1mm]
		\noindent {\em Keywords:} 
	\end{abstract}
	
	\section{Introduction}
	
	\subsection{Novelties}

	New compact fourth-order scheme for Maxwell-equations in $3D$ using equation-based method in staggered grid.
	The staggered structure impose several difficulties in the analysis which enforce us to pick boundary conditions so that the energy estimates will hold.

	
	\subsection{Maxwell equations}
	Let $\e,\h,\bf{D},\bf{B},\bf{J}$ be (real-valued) vector fields defined in  $[0,\infty)\times\Omega$ and take values in $\R^3$.
	The Maxwell equations (without polarization), in first order differential vector form, are given by
	\begin{align*}
		\frac{\partial \bf{D}}{\partial t} &= \nabla \!\times\! \bf{H} - \bf{J} &
		\frac{\partial B}{\partial t}=-\nabla\times \e &\\
		\bf{B} &= \mu \bf{H} & \bf{D} = \epsilon \bf{E},
	\end{align*}
	with initial conditions given for $\bf{D}$ and $\bf{E}$ at $t=0$.

We assume the following:
\begin{enumerate}
	\item  The functions $\mu,\varepsilon$ are    positive constants satisfying $\frac{1}{\mu \varepsilon}=c^2$.
	\item  $\Omega=[0,1]^3$, $\vec{n} \times {\bf E}=0$ on $\partial \Omega$ (\cite[Section 8]{rolf}).
\end{enumerate}
In fact, the identity $\nabla\cdot (\nabla\times)=0$ implies that one can look for solutions  satisfying $\nabla\cdot \h=0$.
By letting  $ \tau=ct$,  $Z=\sqrt{\mu/\varepsilon}$, we look for a numerical scheme for the following problem.
%$$
%\frac{\partial {\bf E}}{\partial \tau }=Z(\nabla \times \bf{H}-{\bf J}), \quad 
%\frac{\partial {\bf H}}{\partial \tau }=-\frac{1}{Z}\nabla \times {\bf E}.
%$$
	\begin{align}
	\label{eq:maxwellv}
	&\notag
\frac{\partial {\bf E}}{\partial \tau }=Z(\nabla \times \bf{H}-{\bf J}) \quad \mathrm{in} \quad  (0,\infty)\times \Omega,\\&\notag
 \frac{\partial {\bf H}}{\partial \tau }=-\frac{1}{Z}\nabla \times {\bf E} \quad \mathrm{in} \quad  (0,\infty)\times \Omega,\\&\notag
 \e(0,x,y,z)=\e_0 \\& \h(0,x,y,z)=\h_0,\\&\notag
	\nabla\cdot \h(\tau,x,y,z) = 0.\notag
\end{align}




	%\begin{remark}
	%	The invariance of the curl operator to rotations imply that
	%our scheme  can be easily generalized to the  case where $\varepsilon$ and $\mu$ are  uniformly   positive definite matrix functions (cf. \cite[Section 8]{rolf}).
	%\end{remark}
	\section{preliminaries}
	We consider uniform discretization of $\Gw=[0,1]^3$ that is,
	 $\Delta x=\Delta y=\Delta z=h$,
	 $\e_h, \h_h$ correspond to discretized functions corresponding $\e,\h$.
	 From time to time, when there is no danger of ambiguity, we omit the subscript $h$ from $\e_h,\h_h$.
	 We consider $\Delta \tau $ as the time-step.
	Moreover, we   introduce the following notations:
	\begin{itemize}
		\item $\e=(E_x,E_y,E_z)$, $\h=(H_x,H_y,H_z)$
		\item $\e^n$  (resp. $\e^{n+1/2}$) denotes $\e(t=n\Delta \tau)$ (resp. $\e^{(n+1/2)(\Delta \tau}$).
		\item $x_{i}=i h,y_{j}=j h,z_{k}=k h$ 
		\item $x_{i+\frac{1}{2}}=\left (i+\frac{1}{2}\right ) h, y_{j+\frac{1}{2}}=\left (j+\frac{1}{2}\right ) h, z_{k+\frac{1}{2}}=\left (k+\frac{1}{2}\right ) h$ 
		\item 	$[N]=\{0,1,..,N\},\quad N\in \N.$
				\item 	$[k,m]=\{k,k+1,..m\},\quad k<m\in \N.$
	\end{itemize}

	To discretize the equations we introduce a staggered mesh in both space and time as in the Yee scheme \cite{yee}. 
	$\e$ is  evaluated at time $n\Delta \tau$ while $\bf{H}$ and $\bf{J}$ are evaluated at time $(n\!+\! \frac{1}{2})\Delta \tau$.

	
	With this arrangement all space derivatives are spread over a single mesh width and the central time and space derivatives
	are centered at the same point similar to that of the Yee scheme \cite{yee}. 
	We define the following meshes:
	\begin{align*}
		&\Gw^{E_z}:=\{(x_i,y_j,z_{k+\frac{1}{2}}),\quad (i,j,k)\in [N]^2 \times[N-1]\} \\
		& \Gw^{E_y}:=\{(x_i,y_{j+\frac{1}{2}},z_{k}), \quad (i,j,k)\in [N]\times [N-1] \times[N] \}\\
		& \Gw^{E_x}:=\{ (x_{i+\frac{1}{2}},y_{j},z_{k}),\quad (i,j,k) \in [N-1]\times [N]^2 \} \\
		&\Gw^{H_z}:=\{ (x_{i+\frac{1}{2}},y_{j+\frac{1}{2}},z_{k}),\quad (i,j,k)\in [N-1]^2\times[N] \}\\
		&\Gw^{H_y}:=\{ (x_{i+\frac{1}{2}},y_j,z_{k+\frac{1}{2}}),\quad (i,j,k) \in [N-1]\times [N] \times [N-1] \}\\
		& \Gw^{H_x}:=\{ (x_i,y_{j+\frac{1}{2}},z_{k+\frac{1}{2}}), \quad (i,j,k) \in [N]\times [N-1]^2\}.
	\end{align*}
Furthermore,
	\begin{align*}
	&\mathrm{int}(\Gw^{E_z}):=\{(x_i,y_j,z_{k+\frac{1}{2}}),\quad (i,j,k)\in [1,N-1]^2 \times[1,N-2]\} \\
	&\mathrm{int}( \Gw^{E_y}):=\{(x_i,y_{j+\frac{1}{2}},z_{k}), \quad (i,j,k)\in [1,N-1]\times [1,N-2] \times[N] \}\\
	& \mathrm{int}(\Gw^{E_x}):=\{ (x_{i+\frac{1}{2}},y_{j},z_{k}),\quad (i,j,k) \in [1,N-2]\times [1,N-1]^2 \} \\
	&\mathrm{int}(\Gw^{H_z}):=\{ (x_{i+\frac{1}{2}},y_{j+\frac{1}{2}},z_{k}),\quad (i,j,k)\in [1,N-2]^2\times[1,N-1] \}\\
	&\mathrm{int}(\Gw^{H_y}):=\{ (x_{i+\frac{1}{2}},y_j,z_{k+\frac{1}{2}}),\quad (i,j,k) \in [1,N-2]\times [1,N-1] \times [1,N-2] \}\\
	& \mathrm{int}(\Gw^{H_x}):=\{ (x_i,y_{j+\frac{1}{2}},z_{k+\frac{1}{2}}), \quad (i,j,k) \in [1,N-1]\times [1,N-2]^2\}.
\end{align*}
By convection, $x_0=y_0=z_0=0$ and $x_N=y_N=z_N=1$.
	We denote 
$$
\Gw^{{\bf E}}:=\Gw^{E_x}\times \Gw^{E_y}\times \Gw^{E_z}, \qquad 
\Gw^{\bf H}:=
\Gw^{H_x}\times \Gw^{H_y}\times \Gw^{H_z},
$$
and 
$$
\mathrm{int}(\Gw^{{\e}}):=\mathrm{int}(\Gw^{E_x})\times \mathrm{int}(\Gw^{E_y})\times \mathrm{int}(\Gw^{E_z}),
\quad \partial \Gw^\e:=\Gw^\e\setminus  \mathrm{int}(\Gw^{\e})
$$
$$
\mathrm{int}(\Gw^{\h}):=
\mathrm{int}(\Gw^{H_x})\times \mathrm{int}(\Gw^{H_y})\times \mathrm{int}(\Gw^{H_z}),
\quad \partial \Gw^\h:=\Gw^\h\setminus  \mathrm{int}(\Gw^{\h})
$$
see Figures \ref{fig:E},\ref{fig:Hx} for illustration of the sets 
$\Gw^{E_z}, \Gw^{H_x},\Gw^{H_y}$ projected on the $x,y$ plane.
{\bf We emphasize that $\partial(\Gw^\h)$ include points which are not interior points
	 in the physical domain $\Gw$	 (in the usual topology on $\R^3$). }


    The dicretized functions $\e_h, \h_h$ are then defined as follows:
    $$
    \e_h^n:=\e(n\Delta \tau,\Gw^\e),\quad  \h_h^{n+\frac{1}{2}}:=\h\left ((n+\frac{1}{2})\Delta \tau,\Gw^\h)\right).
    $$

	With this arrangement, the boundary condition 
	$\vec{n}\times {\bf E}=0$ on $\partial \Gw^{{\bf E}}$ readily imply that 
	$E_s=0$ on $\partial \Gw^{E_s}$ for $s=x,y,z$. That is, 
	\begin{align*}
		&
		E_x(x_{i+\frac{1}{2}},0,z_j)=E_x(x_{i+\frac{1}{2}},N,z_j)=
		E_x(x_{i+\frac{1}{2}},y_j,0)=E_x(x_{i+\frac{1}{2}},y_j, N)=0, \\&
		E_y(0,y_{j+\frac{1}{2}},z_j)=E_y(N,y_{j+\frac{1}{2}},z_j)=
		E_y(x_i,y_{j+\frac{1}{2}},0)=E_y(x_i, y_{j+\frac{1}{2}}, N)=0, \\&
		E_z(0,y_j,z_{k+\frac{1}{2}})=E_x(N,y_j,z_{k+\frac{1}{2}})=
		E_z(x_i,0,z_{k+\frac{1}{2}})=E_z(x_i,N,z_{k+\frac{1}{2}})=0. &
	\end{align*} 
	We define the following operators for subsequent use.
	
	Next, we consider fourth-order approximation for first derivatives:
	Assume that $f(x)$ is known at N points $x_0,x_1, ..,x_{N-1}$. Then,  one can  estimate  $f'(x+h/2)$ at the $N-1$ points 	$x_{\frac{1}{2}},..,x_{N-3/2}$ to fourth order.
   That is, there exists operators $a$ satisfying $a^{-1}D_xf(x)=f'(x+h/2)+O(h^4)$
   (see Appendix \ref{appendix}). For partial derivatives with respect to $s$ we write $a_s$ as the corresponding operator for evaluating $f_s$. 
   The operator $a$ can be modified to a symmetric operator such that fourth-order accuracy is preserved except at the points $x_{\frac{1}{2}}, x_{\frac{N-3}{2}}$.
   In such a case $a$ will be replaced by $a^{\mathrm{sym}}$ (see \eqref{eq:lhs2}).
	\begin{itemize}
		\item $r=\Delta \tau/h$.
		\item ${\bf x'}=(x_i',y_j',z_k')\in \Gw^\e\cup\Gw^\h$
		\item $\delta_{\tau} U^{n+\frac{1}{2}}:=\frac{U^{n+1}-U^{n}}{\Delta \tau}$
				\item $D_x U({\bf x'}):=\frac{U(x_i'+h/2,y_j',z_k')-U(x_i'-h/2,y_j',z_k')}{h}$
	\item $D_y U({\bf x'}):=\frac{U(x_i',y_j'+h/2,z_k')-U(x_i'-h/2,y_j',z_k')}{h}$
		\item $D_z U({\bf x'}):=\frac{U(x_i',y_j',z_k'+h/2)-U(x_i'-h/2,y_j',z_k')}{h}$
%        \item $a_xU({\bf x'}):=\frac{U(x_i'-h/2,y_j',z_k')+22\cdot	 U(x_i',y_j',z_k')+U(x_i'+h/2,y_j',z_k')}{24}$.
%                \item $a_yU({\bf x'}):=\frac{U(x_i',y_j'-h/2,z_k')+22\cdot U(x_i',y_j',z_k')+U(x_i',y_j'+h/2,z_k')}{24}$.
%                        \item $a_zU({\bf x'}):=\frac{U(x_i',y_j',z_k'-h/2)+22\cdot U(x_i',y_j',z_k')+U(x_i',y_j',z_k'+h/2)}{24}$.
		\item $\delta_x U({\bf x' }):=a_x^{-1}\circ D_xU({\bf x}')$
		\item $\delta_y U:=a_y^{-1}\circ D_y$
		\item 	$\delta_z U:=a_z^{-1}\circ D_z$
	\end{itemize}
		Consider  the spaces 
		$$
		V^{\bf{E}}:=\{F \in \Gw^{\bf{E}}:\|F\|_{\bf{E}}<\infty\}, \qquad 
		V^{\bf{H}}:=\{G \in \Gw^{\bf{H}}:\|G\|_{\bf{H}}<\infty\}.
		$$
		Let $F,G$ defined in $\Gw^{\bf{E}}$ and $U,V$ defined in 
		$\Gw^{\bf{H}}$. Assume that either $F$ or $G$ vanish on $\partial \Gw^\e$.
		We define the following inner products:
		$$
		(F,G)_{\Gw^\e}:=
		h^3\sum_{i=0}^{N-1}\sum_{j=0}^{N-1}\sum_{k=0}^{N-1}
		F_{x,i+\frac{1}{2},j,k}G_{x,i+\frac{1}{2},j,k}+
		F_{y,i,j+\frac{1}{2},k}G_{y,i,j+\frac{1}{2},k}+F_{z,i,j,k+\frac{1}{2}}G_{z,i,j,k+\frac{1}{2}}
		$$
		\begin{align*}
			&
			(U,V)_{\Gw^\h}:=
			h^3\sum_{i=0}^{N-1}\sum_{j=0}^{N-1}\sum_{k=0}^{N-1}
			F_{x,i,j+\frac{1}{2},k+\frac{1}{2}}G_{x,i,j+\frac{1}{2},k+\frac{1}{2}}+\\&
			F_{y,i+\frac{1}{2},j,k+\frac{1}{2}}G_{y,i+\frac{1}{2},j,k+\frac{1}{2}}+F_{z,i+\frac{1}{2},j+\frac{1}{2},k}G_{z,i+\frac{1}{2},j+\frac{1}{2},k}.
		\end{align*}
		Since either $F$ or $G$ vanish on $\partial \Gw^\e$, this inner products are the standard inner products defined by elementwise  multiplications in $\Gw^\e, \Gw^\h$.
		Similarly, we define the inner products on $\partial \Gw^\e, \partial \Gw^\h$
		and $\intr{\Gw^\e},\intr{\Gw^\h}$.
		
		
		
		Next, we define 
		$$
		\mathbf{\curl_h}:V^{\bf{E}}\to V^{\bf{H}}, \quad 
		\mathbf{\curl_h'}:V^{\bf{H}}\to V^{\bf{E}},
		$$
		by
		$$
		\mathbf{\curl_h}( resp. \mathbf{\curl_h}')=
		\begin{pmatrix}
			0& -\D_z & \D _y\\
			D_z&0&-D_x\\
			-D_y&D_x&0\\
		\end{pmatrix}.
		$$
		We omit the superscript $'$ where there is no danger of ambiguity.
		
		We notice that 
		\begin{align*}
			&
			\begin{pmatrix}
				0& -\delta_z & \delta _y\\
				\delta_z&0&-\delta_x\\
				-\delta_y&\delta_x&0\\
			\end{pmatrix}=
			\begin{pmatrix}
				0& 0 & 0\\
				0&a_x^{-1}&0\\
				0&0&a_x^{-1}\\
			\end{pmatrix}
			\begin{pmatrix}
				0& 0 & 0\\
				0&0&-D_x\\
				0&D_x&0\\
			\end{pmatrix}+\\&
			\begin{pmatrix}
				a_y^{-1}& 0 & 0\\
				0&0&0\\
				0&0&a_y^{-1}\\
			\end{pmatrix}
			\begin{pmatrix}
				0& 0 & D _y\\
				0&0&0\\
				-D_y&0&0\\
			\end{pmatrix}+
			\begin{pmatrix}
				a_z^{-1}& 0 & 0\\
				0&a_z^{-1}&0\\
				0&0&0\\
			\end{pmatrix}
			\begin{pmatrix}
				0& -D_z & 0\\
				D_z&0&0\\
				0&0&0\\
			\end{pmatrix}.
		\end{align*}
		We then define 
		$$
		<a^{-1},\curl>:=
		\begin{pmatrix}
			0& -\delta_z & \delta _y\\
			\delta_z&0&-\delta_x\\
			-\delta_y&\delta_x&0\\
		\end{pmatrix}.
		$$
		
%		\item $
%		L_{\tau}^h(U):=-
%		\left (
%		\frac{D_{xx}}{1+\frac{h^2}{12}D_{xx}}+\frac{D_{yy}}{1+\frac{h^2}{12}D_{yy}}
%		+\frac{D_{zz}}{1+\frac{h^2}{12}D_{zz}}\right)U
%		+\frac{24}{\Delta \tau^2}U
%		$
%		\item
%		$D_{ss}\phi=\frac{\phi(s+h)-2\phi(s)+\phi(s-h)}{h^2}$ is standard symmetric second derivative operator for $s=x,y,z$ with Dirichlet boundary conditions.


	
	Finally, we define 
	$$
	P({\bf J^{n+\frac{1}{2}}}):=Z\left ( -{\bf J}^{n+\frac{1}{2}}-\nabla div({\bf J}^{n+\frac{1}{2}})+\frac{\partial^2{\bf J}^{n+\frac{1}{2}}}{\partial t^2}\right ).
	$$
   \subsection{Modified Helmholtz equation}
   In the following sections we will exploit fourth-order compact schemes for solving elliptic equations of the form $-\Delta \phi+k^2\phi=k^2F$,$k>0$.
%   $-\Delta (\phi-F)+k^2(\phi-F)=\Delta F $.
Let 
$$
\Delta_h:=D_{xx}+D_{yy}+D_{zz}, \quad \Delta^2_h:= D_{zz}D_{xx}+D_{yy}D_{zz}+D_{xx}D_{zz}.
$$
then 
\begin{align*}
	&
	-\left (
	\Delta_h+\frac{h^2}{6}\Delta^2_h
	\right)\phi+
	k^2\left (
	1+\frac{k^2h^2}{12}
	\right)\phi=
	k^2\left (
	-1-\frac{k^2h^2}{12}
	-\frac{h^2}{12}\Delta_h
	\right)F
\end{align*}
The latter scheme requires boundary operators for the inhomogeneous term $F$ and for $\phi$. We denote these boundary operators as 
$B_{F}$ and $B_{\phi}$ respectively.


If further $B_{\phi}=TB_{F}=B$, 
The resulting scheme can be written as follows:
$$
P_1\phi+\tilde{B}_{\phi}=P_2F.
$$
where $P_1,P_2$ are symmetric operators and $B_{\phi,F}$ is a boundary term depending on $F$ and $\phi$.
%with homogeneous  boundary conditions for $v$.
%In the following sections we will use the demand $B_{\phi}=B_{F}$ in our scheme and show how it simplifies the energy estimates. Moreover, this requirement is also consistent with Taylor expansion of the equations. 
%%\begin{proposition}
%	Let 
%	$$
%	P_1:=	\Delta_h+\frac{h^2}{6}\Delta^2_h
%	+
%	k^2\left (
%	1+\frac{h^2}{12}\Delta_h
%	+\delta\frac{h^4}{144}\Delta^2_h
%	\right)
%	$$
%	$$
%	P_2:=1+\frac{h^2}{12}\Delta_h
%	+\delta\frac{h^4}{144}\Delta^2_h.
%	$$
%\end{proposition}
   
%   $$
%-   \left (
%   \frac{D_{xx}\phi}{1+\frac{h^2}{12}D_{xx}}+\frac{D_{yy}\phi}{1+\frac{h^2}{12}D_{yy}}
%   +\frac{D_{zz}\phi}{1+\frac{h^2}{12}D_{zz}}\right)
%   +k^2\phi=F
%   $$
%    \cite{singer_turkel}.
%   Equivalently,
%   \begin{align*}
%&
%-\left(1+\frac{h^2}{12}D_{yy}\right)
%\left(1+\frac{h^2}{12}D_{zz}\right)D_{xx}\phi-
%\left(1+\frac{h^2}{12}D_{xx}\right)
%\left(1+\frac{h^2}{12}D_{zz}\right)D_{yy}\phi\\&
%-\left(1+\frac{h^2}{12}D_{xx}\right)
%\left(1+\frac{h^2}{12}D_{yy}\right)D_{zz}\phi+
%\left(1+\frac{h^2}{12}D_{xx}\right)
%\left(1+\frac{h^2}{12}D_{yy}\right)
%\left(1+\frac{h^2}{12}D_{zz}\right)k^2\phi
%=\\&
%\left(1+\frac{h^2}{12}D_{xx}\right)
%\left(1+\frac{h^2}{12}D_{yy}\right)
%\left(1+\frac{h^2}{12}D_{zz}\right)F.
%\end{align*}
%This can be generalized to 
%\begin{align*}
%	&
%	\left (
%	D_{xx}+D_{yy}+D_{zz}+\frac{h^2}{6}(D_{zz}D_{xx}+D_{zz}D_{yy}+D_{xx}D_{zz})
%	+\gamma\frac{h^4}{48}D_{xx}D_{yy}D_{zz}
%	\right)\phi+\\&
%	k^2\left (
%     1+\frac{h^2}{12}(D_{xx}+D_{yy}+D_{zz})	
%     +\delta\frac{h^4}{144}(D_{xx}D_{yy}+D_{yy}D_{zz}+D_{xx}D_{zz})	
%     +\sigma\frac{h^6}{1728}D_{xx}D_{yy}D_{zz}
%	\right)\phi=\\&
%		\left (
%	1+\frac{h^2}{12}(D_{xx}+D_{yy}+D_{zz})	
%	+\delta_2\frac{h^4}{144}(D_{xx}D_{yy}+D_{yy}D_{zz}+D_{xx}D_{zz})	
%	+\sigma_2\frac{h^6}{1728}D_{xx}D_{yy}D_{zz}
%	\right)F
%\end{align*}
%If we set $\sigma=\sigma_2=\gamma=\delta=0$ 

%If  $k^2\phi=F$ on the boundary then the added boundary terms  
%reads as follows:
%\begin{align*}
%	&\phi_B
%\end{align*}


   
	\section{The scheme}
	Using  Taylor series and the Maxwell equations \eqref{eq:maxwellv} we derive the following non-homogeneous elliptic equations for (see Appendix \ref{appendxib})
$	\delta_{\tau} \bf{E}$, and $\delta_{\tau} \bf{H}$.
	$$
	-\Delta\delta_{\tau} \e^{n+\frac{1}{2}}+\frac{24}{\Delta \tau^2}\delta_{\tau}\e^{n+\frac{1}{2}}=
	Z\frac{24}{\Delta \tau^2}\nabla\times \h^{n+\frac{1}{2}}+P(\J)^{n+\frac{1}{2}}+
	\mathrm{O(\Delta \tau^4+h^4)} \quad {\mathrm in} \quad (0,1)^3,
	$$
		$$
	-\Delta \delta_{\tau}\h^{n+1}+\frac{24}{\Delta \tau^2}\delta_{\tau}\h^{n+1}=
	-\frac{1}{Z}\frac{24}{\Delta \tau^2}\nabla\times \e^{n+1}+
	\mathrm{O(\Delta \tau^4+h^4)} \quad {\mathrm in} \quad (0,1)^3.
	$$
	\begin{defi}
		For any $s=x,y,z$ and $I= H_s, E_s,$
		let $$P_{1,h}^{I}, P_{2,h}^{I}:\intr\Gwh^{I}\to \intr\Gwh^{I}$$ be the symmetric operators defined by
		$$
		P_{1,h}^{I}:=-\left (\Delta_h+\frac{h^2}{6}\Delta_h^2\right)+\frac{24}{\Delta \tau^2}\left (
		 1+\frac{2}{r^2},
		\right)
		$$
		$$
			P_{2,h}^{I}:=\frac{24}{\Delta \tau^2}\left (
		-1-\frac{2}{r^2}-\frac{h^2}{12}\Delta_h.
		\right)
		$$	
		
%		Let $$L^I:=
%		\left (\frac{P_1^{I}}{P_2^I}\right)+\left (\frac{P_1^I-P_2^I}{P_2^I}\right)B^I
%		$$
		
	\end{defi}
\begin{rem}\label{lem:bottom_spectrum}
	There exists $\lambda_{\mathrm{crit}}$ such that for all $r<\lambda_{\mathrm{crit}}$
 $P_1$ is positive definite, $P_2$ is non-singular.
Moreover, there exists positive  constants depending on $r$,(and  do not depend on $h$), $C_1,C_2, C_3$, such that
if $r<\lambda_{\mathrm{crit}}$, then
$C_3\leq \frac{\Delta \tau^2 }{24}\|P_2\|\leq C_1$.
and 
$\frac{\Delta \tau^2 }{24}\|P_1\|\geq C_2$.
\end{rem}
%	\begin{rem}		
%		 Appendix \ref{appendxib} also shows how our scheme can be extended to the cases $\varepsilon$ and $\mu$ are positive scalar functions which do not depend on $t$.
%	\end{rem}
	The resulting scheme reads as follows.\\[1mm]
	Let $\bf{E}^n_h, \bf{H}^{n+\frac{1}{2}}_h$  be given at $\Gw^{\e}$ and $\Gw^{\h}$ respectively. \\[2mm]
	{\bf step 1}: solve the following equations in $\intr{\Gw^{\e}}$ with
	 $B^{E_s^*}=0$ for $s=x,y,z$: \\[2mm]
	$$
	\tilde{\e}_h^{n+1}:=
	\left.
	\begin{pmatrix}
		\tilde{E}_x^{n+1}\\
		\tilde{E}_y^{n+1}\\
		\tilde{E}_z^{n+1}
	\end{pmatrix}\right|_{\; \intr{\Gw^{\e}}}^{\;}=
\left.
	\begin{pmatrix}
		E_x^n+\Delta \tau E_x^{*}\\
		E_y^n+\Delta \tau E_y^{*}\\
		E_y^n+\Delta \tau E_y^{*}
	\end{pmatrix}\right|_{\; \intr{\Gw^{\e}}}^{\;}
	$$
	where 
	$$
	\underbrace{
	\begin{pmatrix}
		P_1^{E_x} & 0&0 \\
		0 & P_1^{E_y} &\\
		0&0&P_1^{E_z} 
	\end{pmatrix}
}_{:=P_1^\e}
	\begin{pmatrix}
		E_x^{*}\\
		E_y^{*} \\
		E_z^{*}
	\end{pmatrix}+\underbrace{B^{n+1}}_{=0}=Z
\left [ \left.
	\underbrace{
	\begin{pmatrix}
		P_2^{E_x} & 0&0 \\
		0 & P_2^{E_y} &\\
		0&0&P_2^{E_z} 
	\end{pmatrix}
}_{:=P_2^\e}
	\begin{pmatrix}
		0& -\delta_z & \delta _y\\
		\delta_z&0&-\delta_x\\
		-\delta_y&\delta_x&0
	\end{pmatrix}
	\begin{pmatrix}
		H_x^{n+\frac{1}{2}}\\
		H_y^{n+\frac{1}{2}}\\
		H_z^{n+\frac{1}{2}}
	\end{pmatrix}+P({\bf J})\right]\right|_{\; \intr{\Gw^{\h}}}^{\;}.
	$$
	Extend $\tilde{\e}_h^{n+1}$
to $\Gw^\e$ using the boundary conditions $B^{n+1}=0$ and derive 
$\e_h^{n+1}$ in $\Gw^\e$. 
We denote this extension as $\e_h^{n+1}=(0,\tilde{\e_h^{n+1}})\in 
\intr{\Gw^\e}\times \partial \Gw^{\e}.
$ 
	\\[1mm]
	{\bf step 2}: solve $\bf{H}^{n+3/2}_h$ in $\intr\Gw^\h$: \\[2mm]
	Consider the boundary conditions
	\begin{equation}\label{eq:BH}
	\begin{pmatrix}
		B^{H_x}\\ B^{H_y}\\ B^{H_z}
	\end{pmatrix}
	=
	-\frac{1}{Z}
	\left[ \left.	
	\begin{pmatrix}
		0& -\delta_z & \delta _y\\
		\delta_z&0&-\delta_x\\
		-\delta_y&\delta_x&0\\
	\end{pmatrix}
	\begin{pmatrix}
		E_x^{n+1}\\
		E_y^{n+1}\\
		E_z^{n+1}
	\end{pmatrix}\right]\right|_{\; \partial{\Gw^{\h}}}^{\;}
	\end{equation}
	(see Remark \ref{rem:expB}).
Solve
	$$
	\tilde{\h}_h^{n+3/2}:=
	\left.
	\begin{pmatrix}
		\tilde{H}_x^{n+3/2}\\
		\tilde{H}_y^{n+3/2}\\
		\tilde{H}_z^{n+3/2}
	\end{pmatrix}\right|_{\; \intr{\Gw^{\h}}}^{\;}=
\left.
	\begin{pmatrix}
		H_x^{n+\frac{1}{2}}+\Delta \tau H_x^{*}\\
		H_y^{n+\frac{1}{2}}+\Delta \tau H_y^{*}\\
		H_y^{n+\frac{1}{2}}+\Delta \tau H_y^{*}
	\end{pmatrix}\right|_{\; \intr{\Gw^{\h}}}^{\;}
	$$
	where 
	$$
	\underbrace{
	\begin{pmatrix}
		P_1^{H_x}  & 0&0 \\
		0 & 	P_1^{H_y}  &\\
		0&0&	P_1^{H_z} 
	\end{pmatrix}
}_{:=P_1^\h}
	\begin{pmatrix}
		H_x^{*}\\
		H_y^{*} \\
		H_z^{*}
	\end{pmatrix}+B^{n+3/2}=-\frac{1}{Z}
\left[ \left.
	\underbrace{
	\begin{pmatrix}
		P_2^{H_x} & 0&0 \\
		0 & P_2^{H_y} &\\
		0&0&P_2^{H_z} 
	\end{pmatrix}
}_{:=P_2^\h}
	\begin{pmatrix}
		0& -\delta_z & \delta _y\\
		\delta_z&0&-\delta_x\\
		-\delta_y&\delta_x&0\\
	\end{pmatrix}
	\begin{pmatrix}
		E_x^{n+1}\\
		E_y^{n+1}\\
		E_z^{n+1}
	\end{pmatrix}\right]\right|_{\; \intr{\Gw^{\h}}}^{\;},
	$$
	and the same boundary conditions are imposed for both 
		Extend $\tilde{\h}_h^{n+3/2}$
	to $\Gw^\h$ using the boundary conditions \eqref{eq:BH} and derive $\h_h^{n+3/2}$.
%	in $\Gw^{\bf{H}}$. Furthermore,
%	$B_{NH}$ equals to the non-homogeneous term which has calculated at the previous step in $\Gw^{\bf{H}}$ including the boundary.
\begin{rem}\label{rem:expB}


Consider the equation based  Taylor expansion
	$$
	\frac{\h^{n+3/2}-\h^{n+\frac{1}{2} } }{\Delta \tau}=
	\frac{\partial \h}{\partial t}+O(\Delta \tau ^2)=
	-\frac{1}{Z}	\begin{pmatrix}
		0& -\delta_z & \delta _y\\
		\delta_z&0&-\delta_x\\
		-\delta_y&\delta_x&0\\
	\end{pmatrix}
	\begin{pmatrix}
		E_x^{n+1}\\
		E_y^{n+1}\\
		E_z^{n+1}
	\end{pmatrix}+O(\Delta \tau^2+h^4).
	$$
	Hence the  boundary conditions \eqref{eq:BH} are 
 indeed consistent with the condition 
	$B_\phi=B_F$ (cf. \eqref{eq:expB}).
 This boundary condition is of order $\Delta \tau^2$.
\end{rem}
	\begin{cor}
		For all points in $\intr \Gw^{\e}\times\intr\Gw^\h$ the scheme is of order $\Delta \tau^4+h^4$.
	\end{cor}
   
	\begin{rem}
	Approximation of the boundary conditions to order $\Delta \tau ^4$ is also possible as depicted in the example in section \ref{sec:TE}. However, we choose the boundary conditions to fit the stability analysis in the subsequent sections.
	\end{rem}

	


	

	\section{Energy estimates}
	The goal of the the current section is proving the following stability result.
	\begin{lemma}\label{lem:coer}
		For any $Z$ and smooth function $\J$ in $[0,1]^3\times [0,\infty)$, there exists 
		$\lambda_{\mathrm{crit}}$ such that if $r\leq \lambda_{\mathrm{crit}}$ then
 the scheme is stable
%. In particular, 
%	     if   $Z=1$
%		%If 	$$\frac{\Delta \tau}{h}\leq \frac{5}{6\sqrt{3}}$$
%	  the scheme is stable provided that 
%		$$
%			1-\frac{\sqrt{3}r}{2}\|
%	a^{-1}+a^{-t}\|\geq 0.
%		$$
		\end{lemma}
%	\begin{lem}[Abel transformation]
%		Let $p\geq 1$ be integer and $\{a_k\}_{k=1}^{p}$,$\{b_k\}_{k=1}^{p}$ be two sequences. Then
%		$$
%		\sum_{k=1}^{p}a_kb_k=a_pB_p-\sum_{k=1}^{p-1}(a_{k+1}-a_k)B_k
%		$$
%		where 
%		$B_k:=\sum_{i=1}^{k}b_i$.
%	\end{lem}
\begin{lem}[\cite{Morton}]
Let $\{a_1\}_{k=0}^{p-1}$,$\{b_k\}_{k=0}^{p}$ be two sequences. Then
$$
\sum_{k=1}^{p}a_{k-1}(b_k-b_{k-1})=a_{p-1}b_p-a_{0}b_0-\sum_{k=1}^{p-1}(a_{k}-a_{k-1})b_k.
$$
\end{lem}



\begin{lemma}(cf. \cite[(37)]{sakka} for the case $a=1$)\label{lem:sym}
	Assume that $\e_h\in \Gw^\e$ and $\h_h\in \Gw^\h$, and
	$\bf{E}_h$ vanishing on $\partial \Gw^{\bf{E}}$. Then,
	\begin{itemize}
		\item  
		$
		(\curl_h\e_h,\h_h)_{\Gw^\h}=
		(\curl_h\h_h,\e_h)_{\Gw^\e}.
		$\\[1mm]
		\item $(\curl _h\e_h,\h_h)_{\Gw^\h}\leq
		  \frac{2\sqrt{3}}{h}\|\e_h\|_{\Gw^\e}
		\|\h_h\|_{\Gw^\h}
		$.

	\end{itemize}

\end{lemma}
%Let $P$ be the operator
%$$
%	P(\phi):=
%		-	\left (
%	\frac{D_{xx}\phi}{1+\frac{h^2}{12}D_{xx}}+\frac{D_{yy}\phi}{1+\frac{h^2}{12}D_{yy}}
%	+\frac{D_{zz}\phi}{1+\frac{h^2}{12}D_{zz}}\right)
%	+\frac{24}{\Delta \tau^2}\phi.
%	$$
%	We define  the non-negattive operator
%	$$
%	\p:V^{\bf{E}}\to V^{\bf{E}}
%	$$
%	as $$\p:=
%	\frac{\Delta \tau^2}{24}
%	\begin{pmatrix}
%		P & 0&0 \\
%		0&P& 0 \\
%		0&0 &P
%	\end{pmatrix}.
%	$$
%    
%	Then for all $f,g\in \Gw^{\bf{E}}$:
%	\begin{enumerate}
%		\item $(\p g,f)_{\bf{E}}=(\bf{P},g)_{\bf{E}}$.
%		\item $({\bf P}f,f)\geq 0$. 
%	\end{enumerate}
%By a similar way we can define  
%	$$
%\bf{P}':V^{\bf{H}}\to V^{\bf{H}}.
%$$
%\begin{defi}
%	For all $\e^n,\e^m\in V^{\e}$ and $\h^n,\h^m\in V^{\h}$  we define 
%	$$
%	\langle
%\e^n,\e^m
%\rangle_\p	:=(\p\e^n,\e^m), \quad 
%	\langle
%\h^n,\h^m
%\rangle_{\p'}	:=(\p'\h^n,\h^m)_\h.
%	$$
%	The induced norms are denoted as $\|\cdot \|_{\p}$ and $\|\cdot\|_{\p'}$.
%\end{defi}
The following Lemma estimate the anti-symmetric part of the operator $a_s^{-1}$.
\begin{lem}\label{lem:antiestimates}
	There exists $C>0$ not depending on $h$ such that $$\|a_s^{-1}-a_s^{-t}\|\leq Ch$$.
\end{lem}
\begin{proof}
	Recall that the matrix  $a_s $ in Appendix \ref{appendix} are not-singular,  $a_s^{-1}-a_s^{-t}$ is anti-symmetric, and $\|a_s\|$ is bounded from above and below for all $s=x,y,z$ uniformly for $h$. Therefore it is enough to show that $a_s^{-1}-a_s^{-t}$ is of constant rank which do not depend on $h$. This follows  by the observation that 
	$\mathrm{rank}(a-a^t)\leq 8$ and for any non-singular matrix and $v\in \mathrm{ker}(A-A^t)$,
	$$
	Av=A^tv \Rightarrow v=A^{-1}A^{t}v=A^{-t}A^{t}v\Rightarrow
	\mathrm{rank}(A^{-1}-A^{-t})\leq 	\mathrm{rank}(A-A^t).
	$$
\end{proof}

\begin{defi}
	We define the following quantities.
	\begin{align*}
	&
	\mathcal{E}_1^{n}:=
	(P_1\tilde{\e}_h^{n},\tilde{\e}_h^{n})_{\intr{\Gw^\e}}+
%	-\Delta \tau\frac{1}{2} (P_2<a^{-1}+a^{-t},\curl_h>\h_h^{n+1/2},\e_h^n)+\\&
	(P_1\tilde{\h}_h^{n+1/2},\tilde{\h}_h^{n+1/2})_{\intr{\Gw^\h}}, \\&
	\mathcal{E}_2^{n}:=-\frac{\Delta \tau}{2}(P_2<a^{-1}+a^{-t}>\curl_h \e^{n},\h^{n+1/2})_{\intr{\Gw^{h}}}\\&
		\mathcal{E}_3^{n}:=\frac{1}{2}(P_{2,ext}\h^{n+1/2},\h^{n+1/2})_{\partial \Gw^\h}\\&
	\mathcal{E}_4^{n}:=\frac{1}{2}((B+B^t)\h^{n+1/2},\h^{n+1/2})
%	+
%	(\h_h^{n+1/2},\h_h^{n+1/2})_{\partial \Gw^{\h}}.
\end{align*}
\end{defi}
The following lemma is well-known if $a_s$ are identity operators for $s=x,y,z$ \cite[Theorem 4.1]{sakka}. It follows  from the symmetry of the operators $\curl,P_1,P_2$. 
\begin{lem}
	Let $C_1(\lambda_{\mathrm{crit}})$,$C_2(\lambda_{\mathrm{crit}})$ and assume that  $r<\lambda_{\mathrm{crit}}$  (see Remark \ref{lem:bottom_spectrum}).
	Then
	\begin{itemize}
		\item $\mathcal{E}^{n}\geq 
		\left(
		\frac{C_2}{C_1}-\frac{\sqrt{3}r}{2}\|
		a^{-1}+a^{-t}\|
		\right)
		\left (\|\e^{n}_h\|^2+\|\h^{n+1/2}_h\|^2 \right)$
		\item 	$
		{\mathcal E}^{n+1}=	{\mathcal E}^{n}+
	\Delta \tau\frac{1}{2}(P_2<a^{-1}-a^{-t},\curl_h>\e_h^{n+1},\h_h^{n+1/2}).
	$
		\end{itemize}
\end{lem}
\begin{proof}
	The first of the proof follows by proposition \ref{lem:bottom_spectrum} and Young's inequality.
	
Next, we write the following scheme equalities in the space $\intr\Gwh$.
	
	\begin{align}
	P_1(\tilde{\e}_h^{n+1} -\tilde{\e}^{n})=
		\Delta \tau P_2 \acurl \h_h^{n+1/2}
	\end{align}
		\begin{align}
			P_1(\tilde{\h}_h^{n+3/2} -\tilde{\h}^{n+1/2})+
			B^{n+3/2}(\h_h^{n+3/2} -\h^{n+1/2})
=
			-\Delta \tau P_2 \acurl \e_h^{n+1}
	\end{align}
.

	Next, we take inner products of the scheme with $\tilde{\e}_h^{n+1}+\tilde{\e}_h^n$,
	and with 
	$\tilde{\h}_h^{n+3/2}+\tilde{\h}_h^{n+1/2}$. Therefore
\begin{align*}
	&
	(P_1(\tilde{\e}^{n+1}_h-\tilde{\e}^{n}_h),\tilde{\e}^{n+1}_h+\tilde{\e}^{n}_h)_{\intr{\Gw^\e}}+
	(P_1(\tilde{\h}^{n+3/2}_h-\tilde{\h}^{n+1/2}_h),\tilde{\h}^{n+3/2}_h+\tilde{\h}^{n+1/2}_h)_{\intr{\Gw^\h}}=\\&
=\mathcal{E}_1^{n+1}-\mathcal{E}_1^{n}.
\end{align*}
Next we estimate
\begin{align*}
	&
		(\frac{1}{2}\Delta \tau P_2<a^{-1}+a^{-t},\curl_h>\h^{n+1/2},\tilde{\e}^{n+1}+\tilde{\e}^{n})_{ \intr{\Gw^\e}}=\\&
				(\frac{1}{2}\Delta \tau P_{2,ext}<a^{-1}+a^{-t},\curl_h>(\e^{n+1}+\e^{n}),\h^{n+1/2})_{ {\Gw^\e}}\\&
								-(\frac{1}{2}\Delta \tau P_{2,ext}<a^{-1}+a^{-t},\curl_h>(\h^{n+1/2}),\e^{n+1}+\e^{n})_{ {\partial{\Gw^\e}}}
\end{align*}
Similarly,
\begin{align*}
	&
	(\frac{1}{2}\Delta \tau P_2<a^{-1}+a^{-t},\curl_h>\e^{n+1},\tilde{\h}^{h+3/2}+\tilde{\h}^{n+1/2})_{\intr{\Gw^\h}}=\\&
		(\frac{1}{2}\Delta \tau P_{2,ext}<a^{-1}+a^{-t},\curl_h>\e^{n+1},\h^{h+3/2}+\h^{n+1/2})_{{\Gw^\h}}-
		\\&
			(\frac{1}{2}\Delta \tau P_{2,ext}<a^{-1}+a^{-t},\curl_h>\e^{n+1},\h^{h+3/2}+\h^{n+1/2})_{\partial{\Gw^\h}}=\\&
			(\frac{1}{2}\Delta \tau P_{2,ext}<a^{-1}+a^{-t},\curl_h>(\h^{h+3/2}+\h^{n+1/2}),\e^{n+1})_{{{\Gw^\e}}}
-\\&
(\frac{1}{2}\Delta \tau P_{2,ext}<a^{-1}+a^{-t},\curl_h>\e^{n+1},\h^{n+3/2}+\h^{n+1/2})_{\partial{\Gw^\h}}.
\end{align*}
Recall that we chose our boundary conditions such that 
the last term equals $\mathcal{E}_3^{n+1}-\mathcal{E}_3^{n}$.
Together with Lemma \ref{lem:antiestimates} we have the following:
\begin{cor}
$$\sum_{i=1}^3\mathcal{E}_i^n-
\sum_{i=1}^3\mathcal{E}_i^{n+1}=
((B^{n+3/2})(\h^{n+3/2}-\h^{n+1/2}),\tilde{\h}^{n+3/2}+\tilde{\h}^{n+1/2})_{\intr{\Gw^\h}}+\mathrm{term}
$$
where 
$$
\mathrm{term}\leq C \Delta \tau( \sum_{i=1}^3\mathcal{E}_i^n+
\sum_{i=1}^3\mathcal{E}_i^{n+1}).
$$
\end{cor}
Without the boundary term $B$, we could prove stability.

To prove stability we need to  estimate  the term 
$$
A:=
\frac{1}{2}((B+B^t)(\h^{n+3/2}-\h^{n+1/2}),\tilde{\h}^{n+3/2}+\tilde{\h}^{n+1/2})_{\intr{\Gw^\h}}.
$$
as we did before.
Unfortunately, 
this term not equals 
$$
\frac{1}{2}((B+B^t)(\widetilde{\h}^{n+3/2}-\tilde{\h}^{n+1/2}),\tilde{\h}^{n+3/2}+\tilde{\h}^{n+1/2})_{\intr{\Gw^\h}}
$$
and here I am stuck.

	
	
%\begin{align*}
%	&
%	(P_1(\e^{n+1}_h-\e^{n}_h),\e^{n+1}_h+\e^{n}_h)_{\intr{\Gw^\e}}
%	=
%	\Delta \tau(P_2\acurl\h_h^{n+1/2},
%	\e^{n+1}_h+\e^{n}_h)_{\intr{\Gw^\e}}
%\end{align*}
%and
%\begin{align*}
%	&
%		(P_1(\h^{n+3/2}_h-\h^{n+1/2}_h),\h^{n+3/2}_h+\h^{n+1/2}_h)_{\intr{\Gw^\h}}
%	= 
%		-\Delta \tau(P_2\acurl\e_h^{n+1},
%	\h^{n+3/2}_h+\h^{n+1/2}_h)_{\intr{{\Gw^\h}}}.
%\end{align*}
%Since $P_1$ is symmetric we have 

\end{proof}

%\begin{cor}
%	There exists 
%	$\lambda_{\mathrm{crit}}$ such that if $r\leq \lambda_{\mathrm{crit}}$
%	the exists $C>0$ which do not depend on $h$ satisfying
%	then 
%	$$
%	\frac{1-C\Delta \tau}{\|P_2\|} \mathcal{E}^{n+1} \leq \frac{1+C\Delta \tau}{\|P_2\|}\mathcal{E}^n.
%	$$
% In particular the scheme is stable.
%\end{cor}

%\begin{rem}
%	For splitting schemes which are based on decomposition of general operators for symmetric and anti-symmetric parts see \cite{....}.
%\end{rem}

%Let $A:=<a^{-1}-a^{-t},\curl_h>$ and notice that $A$ is anti-symmetric.
%\begin{align*}
%	&
%	(A\e_h^{n+1},\h_h^{n+1/2})=\\&
%		\frac{(A\e_h^{n+1},\h_h^{n+1/2})}{2}-	\frac{(A\h_h^{n+1/2},\e_h^{n+1})}{2}=\\&		
%		\frac{(A\e_h^{n+1},\h_h^{n+1/2}+\h_h^{n+3/2}-\h_h^{n+3/2})}{2}
%-\frac{(A\h_h^{n+1/2},\e_h^{n+1}+\e_h^{n}-\e_h^{n})}{2}=\\&
%	\frac{(A\e_h^{n+1},\h_h^{n+3/2})}{2}+
%		\frac{(A\e_h^{n+1},\h_h^{n+1/2}-\h_h^{n+3/2})}{2}\\&
%		-	\frac{(A\h_h^{n+1/2},\e_h^{n})}{2}-
%		\frac{(A\h_h^{n+1/2},\e_h^{n+1}-\e_h^{n})}{2}=\\&
%	\frac{(A\e_h^{n+1},\h_h^{n+3/2})}{2}-\frac{(A\h_h^{n+1/2},\e_h^{n})}{2}\\&
%		+\Delta \tau\frac{(A\e_h^{n+1},\frac{P_2}{P_1}\acurl\e_h^{n+1})}{2}-
%			\Delta \tau \frac{(A\h_h^{n+1/2},\frac{P_2}{P_1}\acurl\h_h^{n+1/2})}{2}
%\end{align*}
%\begin{cor}
%	Let 
%	$$\mathcal{\tilde{E}}^{n}:=\mathcal{E}^n-\Delta \tau 	\frac{(A\e_h^{n},\h_h^{n+1/2})}{2}
%	-
%	\Delta \tau^2\frac{(A\e_h^{n},\frac{P_2}{P_1}a^{-1}\e_h^{n})}{2}
%	$$
%	Then $\mathcal{\tilde{E}}^{n+1}\leq \mathcal{\tilde{E}}^{n}$.
%\end{cor}
%\begin{lem}
%	\begin{itemize}
%		\item 
%	\end{itemize}
%	$$
%	|\mathcal{\tilde{E}}^{n}|\leq 
%	\mathcal{E}^n-r\sqrt{3}\|A\|
%	$$
%\end{lem}
%\begin{lem}
%	$$
%	\xi^{n}:=
%\left | (\frac{P_1-P_2}{P_2}B(\h^{n+3/2}_h-\h^{n+1/2}_h),\h^{n+3/2}_h+\h^{n+1/2}_h)_{\intr{\Gw^\h}}\right|\ \leq ..
%$$
%\end{lem}
%\begin{proof}
%	$\h^{n+3/2}-\h^{n+1/2}=-\Delta \tau \curl_h\e^{n+1}$. By Lemma \ref{lem:sym} and CS inequality,
%	\begin{align*}
%	\xi^n\leq \Delta \tau \left \| \frac{P_1-P_2}{P_2}\right\|\cdot  \frac{2\sqrt{3}}{h}
%\|\e_h^{n+1}\|_{\Gw^\e}\cdot \|\h_h^{n+3/2}+\h_h^{n+1/2}\|_{\Gw^\h}.
%	\end{align*}
%By Young's inequality
%\begin{align*}
%\|\e_h^{n+1}\|_{\Gw^\e}\cdot \|\h_h^{n+3/2}+\h_h^{n+1/2}\|_{\Gw^\h}\leq 
%\frac{\|\e_h^{n+1}\|^2}{2}+
%\|\h_h^{n+3/2}\|^2+
%\|\h_h^{n+1/2}\|^2.
%\end{align*}
%Recall that 
%$$
%\left \| \frac{P_1-P_2}{P_2}\right\|\leq \frac{4\Delta\tau^2}{24h^2}=\frac{\Delta \tau^2}{6h^2}=\frac{r^2}{6}.
%$$
%
%Therefore
%$$
%\mathcal{E}^{n+1}-\frac{2r^3\sqrt{3}}{6}\left ( \frac{\|\e_h^{n+1}\|^2}{2}+
%\|\h_h^{n+3/2}\|^2  \right)\leq 
%\mathcal{E}^{n}+\frac{2r^3\sqrt{3}}{6}\|\h_h^{n+1/2}\|^2 .
%$$
%By Lemma \ref{lem:coer},
%\begin{align*}
%	&
%(\|\e_h^{n+1}\|^2+\|\h_h^{n+3/2}\|^2)
%\left (1-\frac{6\sqrt{3}r}{5}-\frac{2r^3\sqrt{3}}{6} \right)\leq \\&
%(\|\e_h^{n}\|^2+\|\h_h^{n+1/2}\|^2)
%\end{align*}
%
%\end{proof}
%Lemma \ref{lem:sym} and summation imply that 
%\begin{align*}
%	&
%	{\mathcal E}^{n+1}=	{\mathcal E}^{n}-
%	(\frac{P_1-P_2}{P_2}B(\h^{n+3/2}_h-\h^{n+1/2}_h),\h^{n+3/2}_h+\h^{n+1/2}_h)_{\intr{\Gw^\h}}.
%\end{align*}

%The positivity of the energy follows from   \cite[Theorem 4.1]{sakka} and Lemma \ref{lem:sym}:
%$$
%\|\curl _h\e_h\|_{\h}\leq \frac{2\sqrt{3}}{h}, \quad \|a_s\|\geq \frac{5}{6}.
%$$
%  (cf. \cite[Theorem 4.1]{sakka}). By Young's inequality, 
% $$
% {\mathcal E}^{n}\geq 
% \|\e_h^{n}\|^{2}_{\p}+\|\h_h^{n+1/2}\|^{2}_{\p'}-
%\frac{6\sqrt{3}\Delta \tau}{5h}\left ( \|\e_h^{n}\|_{\e}^2+\|\h_h^{n+1/2}\|_{\h}^2 \right).
% $$
%Since $\frac{\Delta \tau^2}{24}\sigma_{\min}(P)\geq 1$ we are done.
%
%
%\begin{lem}
%	Assume for simplicity that $\J=0$ and $Z=1$. Moreover, assume that 
%	$n\Delta \tau\leq T$.
%	\textcolor{red}{Then we can show stability for certain cfl.. ..???}
%	that is $$\mathcal{E}^{n+1}\leq CT\mathcal{E}^{n}$$.
%\end{lem}
%\begin{proof}
%	\begin{align*}
%		&
%			(\p(\e_h^{n+1}-\e_h^n),\e_h^{n+1}+\e_h^n)_\e=
%		\Delta \tau (\curl_h \h^{n+1/2},\e^{n+1}+\e_h^n)_{\e}, \\&
%		(\p'(\h_h^{n+3/2}-\h_h^{n+1/2}),\h_h^{n+3/2}+\h_h^{n+1/2})_\h=
%		-\Delta \tau (\curl_h' \e_h^{n+1},\h_h^{n+3/2}+\h_h^{n+1/2})_{\h}.
%	\end{align*}
%Therefore, 
%$$
% \|\e_h^{n}\|^{2}_{\p}+\|\h_h^{n+1/2}\|^{2}_{\p'}=
% $$
%\end{proof}
%\begin{lem}
%	Assume that $U,V$ satisfy the boundary conditions 
%	$U_{i+\frac{1}{2},0}=U_{i+\frac{1}{2},N}=V(0,j+\frac{1}{2})=V(N,j+\frac{1}{2})=0$.
%	Then
%	$$
%	\sum_{j=0}^{N-1}W_{i+\frac{1}{2},j+\frac{1}{2}}\delta_yU_{i+\frac{1}{2},j+\frac{1}{2}}=
%	-\sum_{j=1}^{N-1}U_{i+\frac{1}{2},j}\delta_y W_{i+\frac{1}{2},j}
%	$$
%\end{lem}
%\begin{lem}
%	Let $E_z(i,j)$, $H_x(i,j+\frac{1}{2})$, $H_y(i+\frac{1}{2},j)$.
%Then
%	\begin{align*}
%		&
%\sum_{i=0}^{N-1}	\sum_{j=0}^{N-1}a_y^{-1}(i,j)\delta_yE_z(i,j+1/2)H_y(i,j+1/2)=
%\sum_{j=0}^{N-1}\delta_yE_z(i,j+1/2)H_y(i,j+1/2)\\&
%-\sum_{i=0}^{N-1}	\sum_{j=0}^{N-1}(\delta_{i,j}-a_y^{-1}(i,j))\delta_yE_z(i,j+1/2)H_y(i,j+1/2)
%	\end{align*}
%\end{lem}
%\begin{cor}
%	\begin{align*}
%		&
%		\left(
%		\begin{pmatrix}
%		0& -\delta_z & \delta _y\\
%		\delta_z&0&-\delta_x\\
%		-\delta_y&\delta_x&0\\
%	\end{pmatrix}
%	\begin{pmatrix}
%		E_x^{n+1}\\
%		E_y^{n+1}\\
%		E_z^{n+1}
%	\end{pmatrix}, 
%\begin{pmatrix}
%	H_x^{n+1}\\
%	H_y^{n+1}\\
%	H_z^{n+1}
%	\end{pmatrix}
%\right)_{\bf{H}}=\\&
%(\mathbf{\curl}(\bf{E}_h),\bf{H}_h)_{\bf{H}}+\Theta^{1}
%	\end{align*}
%and $$\|\Theta^{1}\|\leq \|I-a^{-1}\|\cdot  \|  {\bf \curl (\bf{E}_h)}\|_{\bf{H}}  \cdot 
%\|H\|_{\bf{H}}\leq \frac{2\sqrt{3}}{h}
%\|I-a^{-1}\|\cdot  \|  {(\bf{E}_h)}\|_{\bf{E}}  \cdot 
%\|H\|_{\bf{H}}
%$$
%\end{cor}
%\begin{lem}
%Assume that $\left \| \frac{\theta+\theta^t}{2}-\theta\right \|\leq \varepsilon$.
%Then, 
%$$
%\left \|
%(\Theta_1 \h^{n+1/2},\e^{n+1})_\e
%-(\Theta_1 \e^{n+1},\h^{n+1/2})_\h \right\|\leq 
%$$
%\end{lem}
%\begin{lemma}
%	If one consider anti- symmetric finite-difference operator for first derivative  then the following term is a discrete energy provided that 
%	$$\frac{\Delta \tau}{h}\leq \frac{1}{\sqrt{3}\|A^{-1}\| }$$.
%	\begin{align*}
%		{\mathcal E}^{n}:=\|\e^{n}\|^{2}_{\e}+\|\h^{n+1/2}\|^{2}_{\h}
%		-\Delta \tau \left ( \curl_h \e_h^{n},\h^{n+1/2}\right )_{\h}
%		-\Delta \tau (\Theta_1\e^{n},\h^{n+1/2})_\h.
%	\end{align*}
%	That is, ${\mathcal E}^{n}$  is coercieve and 
%	$$
%		{\mathcal E}^{n+1}\leq 	{\mathcal E}^{n}.
%	$$
%\end{lemma}
%\begin{proof}
%	{\em monotonicity:}
%\begin{align*}
%	&
%	\frac{1}{\Delta \tau}\left [	\|\e_h^{n+1}\|^2-\|\e_h^{n}\|^2+
%		\|\h_h^{n+3/2}\|^2-\|\h_h^{n+1/2}\|^2\right ]=\\&
%		-\left ( \curl_h \e_h^{n+1},\h_h^{n+3/2}\right )
%		+\left ( \curl_h \h_h^{n+1/2},\e_h^{n}\right )+\\&
%		+(\Theta_1 \h^{n+1/2},\e^{n+1})_\e+(\Theta_1\h^{n+1/2},\e^n)_\e\\&
%		-(\Theta_1 \e^{n+1},\h^{n+1/2})_\h-(\Theta_1\e^{n+1},\h^{n+3/2})_\h=\\&
%			-\left ( \curl_h \e_h^{n+1},\h_h^{n+3/2}\right )
%		+\left ( \curl_h \h_h^{n+1/2},\e_h^{n}\right )+\\&
%		(\Theta_1\h^{n+1/2},\e^n)_\e
%-(\Theta_1\e^{n+1},\h^{n+3/2})_\h.
%\end{align*}
%	{\em coercivity:}
%	
%	\begin{align*}
%		&
%	|	\Delta \tau \left ( \curl_h \e_h^{n},\h^{n+1/2}\right )_{\h}|\leq 
%	\Delta \tau 
%    \|\curl\|\|\e_h^n\|\|\h^{n+1/2}\|\leq
%    \\&
%     \Delta \tau \|A^{-1}\|\frac{2\sqrt{3}}{h}\|\e_h^n\| \| \h_h^{n+1/2}\|\leq
%	    \Delta \tau \|A^{-1}\|\frac{\sqrt{3}}{h}
%	    (\|\e_h^n\|_{\e}^2+ \| \h_h^{n+1/2}\|_{\h}^2).
%	\end{align*}
%\end{proof}
%\begin{rem}
%	When $A$ is not symmetric the stability condition is 
%	\begin{align*}
%	{\mathcal E}^{n}:=\|\e^{n}\|^{2}_{\e}+\|\h^{n+1/2}\|^{2}_{\h}
%	-\Delta \tau \left (\tilde{ \curl}_h \e_h^{n},\h^{n+1/2}\right )_{\h}
%\end{align*}
%\end{rem}



	
	
	

%	\subsection{The vaccum case}
%	For constant  $\mu,\varepsilon$ satisfying 
%	$1/(\mu \varepsilon)=c^2$ the scheme can be further simplified to
%	\begin{equation}\label{eq:mu_eps_constant_lap}
%		\delta_t D_z-\frac{c^2\Delta t^2}{24}\Delta ( \delta_t D_z)=
%		\left ( \frac{\partial H_y}{\partial x}-
%		\frac{\partial H_x}{\partial y}-J_z
%		\right)^{n+\frac{1}{2}}+\frac{c^2 \Delta t^2}{24}\left (
%		-
%		\frac{\partial div(\vec{J})}{\partial z}-\frac{\partial^2 J_z}{\partial t^2}
%		\right)^{n+\frac{1}{2}}.
%	\end{equation}
%	Let us define $\Delta \tau=c\Delta t$,  $Z=\sqrt{\mu/\varepsilon}$,
%	$$L_{\tau}(u):=- \frac{\Delta \tau^2}{24}\Delta u+u,$$
%	and finally
%	$$
%	\delta_{\tau} \vec{E}=\frac{\vec{E}^{n+1}-\vec{E}^n}{\Delta \tau}, \qquad
%	\delta_{\tau} \vec{H}=\frac{\vec{H}^{n+\frac{1}{2}}-\vec{H}^{n+3/2}}{\Delta \tau}.
%	$$
%	The full scheme is then given by 
%	\begin{equation}\label{eq:system}
%		\begin{pmatrix}
%			\vec{L}_{\tau} & 0 \\
%			0 & \vec{L}_{\tau}
%		\end{pmatrix}
%		\begin{pmatrix}
%			\delta_{\tau} \vec{E} \\
%			\delta_{\tau} \vec{H}
%		\end{pmatrix}=
%		\begin{pmatrix}
%			\vec{Z}& 0 \\
%			0&	\frac{1}{\vec{Z}}
%		\end{pmatrix}
%		\begin{pmatrix}
%			\nabla \times \vec{H}^{n+\frac{1}{2}} -\vec{J}^{n+\frac{1}{2}}\\
%			-\nabla \times \vec{E}^{n+1}
%		\end{pmatrix}+
%		\frac{\Delta \tau ^2}{24}
%		\begin{pmatrix}
%			\left(-\nabla div(\vec{J})+
%			\frac{\partial^2 \vec{J}}{\partial t^2}\right)^{n+\frac{1}{2}} \\
%			0
%		\end{pmatrix}.
%	\end{equation}
%	The equations in \eqref{eq:system} can be solved in bounded domains to fourth order accuracy using the scheme  in \cite{singer_turkel}. It is important to notice that for bounded domains the boundary conditions for $\delta_t \vec{H}, \delta_t \vec{E}$ has to be approximated to fourth order.
%	
	
	%	+ \frac{\partial^2 {D_x \over \epsilon}}{\partial y \partial t} \right) + O(\Delta t)^4
	%\end{align*}
	%We then substitute the Maxwell equation for $\frac{\partial D}{\partial t}$ and get
	%\begin{equation*}
	%	\frac{\partial B_z}{\partial t} =
	%	\delta_t B_z - \frac{\Delta t^2}{24} \frac{\partial}{\partial t} \left(\frac{\partial^2 {H_z \over \epsilon}}{\partial x^2}  + \frac{\partial^2 {H_z \over \epsilon}}{\partial y^2}
	%	- \frac{\partial^2 {H_x \over \epsilon}}{\partial x \partial z} - \frac{\partial^2 {H_y \over \epsilon}}{\partial y \partial z} + \frac{\partial J_y}{\partial x} - \frac{\partial J_x}{\partial y}\right)
	%	+ O(\Delta t)^4
	%\end{equation*}
	%We now add and subtract $\frac{\partial^2 {H_z \over \epsilon}}{\partial z^2}$ to the terms in the parenthesis and then replace $H$ by $\frac{B}{\mu}$ and then set $c^2 = \frac{1}{\mu \epsilon}$.
	%We note that $c^2$ is constant even when $\epsilon$ and $\mu$ vary.
	%This yields
	%\begin{equation}
	%	\frac{\partial B_z}{\partial t} = \delta_t B_z - \frac{\Delta t^2}{24} \frac{\partial}{\partial t} \left(c^2 \Delta B_z - c^2 \frac{\partial }{\partial z} div \vec{B}
	%	+ \frac{\partial J_y}{\partial x} - \frac{\partial J_x}{\partial y} \right)
	%	+ O(\Delta t)^4
	%	\label{max10}
	%\end{equation}
	%%We further assume that the divergence of both $B$ and $D$ are zero.
	%%If $div(D) = J \neq 0$ then we have an extra term
	%%$\frac{\Delta t^2}{24 c^2} \frac{\partial^2 J}{\partial t \partial z}$,
	%Then \eqref{max10} reduces to
	%\begin{equation}
	%	\frac{\partial B_z}{\partial t} = \delta_t B_z -  \frac{\Delta t^2}{24} \frac{\partial}{\partial t} \left(c^2 \Delta B_z + \frac{\partial J_y}{\partial x} - \frac{\partial J_x}{\partial y} \right)
	%	+ O(\Delta t)^4
	%	\label{max20}
	%\end{equation}
	%Finally, we replace $\frac{\partial \Delta B_z}{\partial t}$ by $\Delta \delta_t B_z +O(\Delta t)^2$.
	%Since it is already multiplied by $(\Delta t)^2$ the total error is $(\Delta t)^4$.
	%So we are left with
	%\begin{equation}
	%	\frac{\partial B_z}{\partial t} = \delta_t B_z - c^2 \frac{\Delta t^2}{24}  \left(\Delta \delta_t B_z + \frac{1}{c^2} \frac{\partial}{\partial t} (\frac{\partial J_y}{\partial x} - \frac{\partial J_x}{\partial y}) \right)
	%	+  O(\Delta t)^4
	%\end{equation}
	% we get
	%\begin{equation}
	%	\label{finalBz}
	%	\delta_t B_z - \frac{c^2 \Delta t^2 }{24}  \Delta (\delta_t B_z)  =
	%	(-\frac{\partial E_y}{\partial x} + \frac{\partial E_x}{\partial y} )^{n+\frac{1}{2}}
	%	+ \frac{\Delta t^2}{24} \frac{\partial}{\partial t} \left( \frac{\partial J_y}{\partial x} - \frac{\partial J_x}{\partial y} \right)
	%	+  O(\Delta t)^4
	%\end{equation}
	%
%	\section{stability analysis}
%	%It seems that $CFL\leq \frac{1}{\sqrt{2}}$ as with Yee.
%	\subsection{The case $\mu, \varepsilon$ are constants and two dimensions}
%	We assume for simplicity that $Z=1$.
%	
%	Let us rewrite  the homogeneous part in   \eqref{eq:system} in two steps:
%	\begin{align*}
%		&
%		\begin{pmatrix}
%			L_{\tau} & 0&0 \\
%			0&L_{\tau} &0 \\
%			0&0&L_{\tau} 
%		\end{pmatrix}
%		\begin{pmatrix}
%			\delta_{\tau} E_x^{n+1} \\
%			\delta_{\tau} E_y^{n+1} \\
%			\delta_{\tau} E_z^{n+1} \\
%		\end{pmatrix}=
%		\frac{1}{h}
%		PD\vec{H}^{n+\frac{1}{2}}
%	\end{align*}
%	\begin{align*}
%		&
%		\begin{pmatrix}
%			L_{\tau} & 0&0 \\
%			0&L_{\tau} &0 \\
%			0&0&L_{\tau}   \\
%		\end{pmatrix}
%		\begin{pmatrix}
%			\delta_{\tau} H_x^{n+3/2} \\
%			\delta_{\tau} H_y^{n+3/2}\\
%			\delta_{\tau} H_z^{n+3/2}
%		\end{pmatrix}=
%		\frac{1}{h}
%		P D^t\vec{E}^{n+1}
%	\end{align*}
%	
%	%
%	%\begin{align*}
%	%	&
%	%	\begin{pmatrix}
%		%		L_{\tau} & 0&0&0&0&0 \\
%		%			0&L_{\tau} &0&0&0&0 \\
%		%				0&0&L_{\tau}  &0&0&0 \\
%		%					0&0&0&L_{\tau} &0&0 \\
%		%						0&0&0&0&L_{\tau} &0 \\
%		%							 0&0&0&0&0&L_{\tau}  \\
%		%	\end{pmatrix}
%	%	\begin{pmatrix}
%		%		\delta_{\tau} E_x^{n+1} \\
%		%			\delta_{\tau} E_y^{n+1} \\
%		%				\delta_{\tau} E_z^{n+1} \\
%		%		\delta_{\tau} H_x^{n+3/2} \\
%		%			\delta_{\tau} H_y^{n+3/2}\\
%		%				\delta_{\tau} H_z^{n+3/2}
%		%	\end{pmatrix}=
%	%\frac{1}{h}
%	%AD\phi
%	%\end{align*}
%	where 
%	$$P=
%	\begin{pmatrix}
%		0 & 1&1 \\
%		1&0 &1 \\
%		1&1&0 
%	\end{pmatrix}\otimes A^{-1}=
%	\begin{pmatrix}
%		0 & A^{-1}&A^{-1} \\
%		A^{-1}&0 &A^{-1} \\
%		A^{-1}&A^{-1}&0 
%	\end{pmatrix},\quad
%	D=
%	\begin{pmatrix}
%		0 & -D_z&D_y \\
%		D_z&0 &-D_x \\
%		-D_y&D_x&0  \\
%	\end{pmatrix}.
%	$$
%	The discretization for $$\Delta \phi -\frac{24}{\tau^2}\phi=\frac{24}{\Delta \tau ^2}F$$ is
%	given by the three-dimensional analog of \cite[(11)]{singer_turkel}
%	$$
%	\frac{D_{xx}\phi}{1+\frac{h^2}{12}D_{xx}}+\frac{D_{yy}\phi}{1+\frac{h^2}{12}D_{yy}}
%	+\frac{D_{zz}\phi}{1+\frac{h^2}{12}D_{zz}}-\frac{24}{\Delta \tau^2}\phi
%	=\frac{24}{\Delta \tau ^2}F.
%	$$
%	We have the following:
%	\begin{enumerate}
%		\item  $|\lambda_{\max}(A^{-1})| \frac{24}{20}$.
%		\item  $|\lambda_{\max}(D)|\leq 2\sqrt{2}$
%	\end{enumerate}
%	
%	
%	The matrices $D_x,D_y,D_z$ are toplitz matrices with two dominant diagonals $1$ and $-1$, and $A^{-1}$ is the inverse of the Pad\'e operator for first derivative in Appendix \ref{appendix}.
%	
	%Then,
	%the discretization for $$\Delta \phi -\frac{24}{\tau^2}\phi=\frac{24}{\Delta \tau ^2}F$$ is
	%given by the three-dimensional analog of \cite[(11)]{singer_turkel}
	%$$
	%\frac{D_{xx}\phi}{1+\frac{h^2}{12}D_{xx}}+\frac{D_{yy}\phi}{1+\frac{h^2}{12}D_{yy}}
	%+\frac{D_{zz}\phi}{1+\frac{h^2}{12}D_{zz}}-\frac{24}{\Delta \tau^2}\phi
	%=\frac{24}{\Delta \tau ^2}F.
	%$$
	
	
	
	
	\section{Example: Transverse magnetic waves  in $[0,1]^3$}\label{sec:TE}
	Assuming that $E_x=E_y=H_z=\vec{J}=0$ we can reduce Maxwell equations to a two-dimensional problem for $E_z(t,x,y), H_x(t,x,y), H_y(t,x,y)$ where  $(x,y)\in[0,1]^2$, and
	$x_0=y_0=0$, and $x_{N-1}=y_{N-1}=0$.
	The rescaled equations are therefore 
	
	\begin{enumerate}
		\item  $ \frac{\partial E_z}{\partial \tau }=Z\left (
		\frac{\partial H_y}{\partial x}-\frac{\partial H_x}{\partial y}
		\right)$
		\item 	  $ \frac{\partial H_x}{\partial \tau }=-\frac{1}{Z}\frac{\partial E_z}{\partial y}$
		\item 	  $ \frac{\partial H_y}{\partial \tau }=\frac{1}{Z}\frac{\partial E_z}{\partial x}$,
		
	\end{enumerate}
	and the scheme  reads as the following consecutive steps.
	\begin{enumerate}
		\item  $ E_z^{n+1}=E_z^n(i,j)+Z\Delta \tau \cdot 
		\left(\frac{P_1^{E_z}}{P_2^{E_z}}\right)^{-1}
		\left (\frac{\partial H_y}{\partial x}-
		\frac{\partial H_x}{\partial y}
		\right)^{n+\frac{1}{2}}$\\[1mm]
		\item $ H_x^{n+3/2}=H_x^{n+\frac{1}{2}}+\frac{\Delta \tau}{Z} \cdot  \left(\frac{P_1^{H_x}}{P_2^{H_x}}\right)^{-1}\left (-
		\frac{\partial E_z}{\partial y}
		\right)^{n+1} $\\[1mm]
		\item $ H_y^{n+3/2}=H_y^{n+\frac{1}{2}}+\frac{\Delta \tau}{Z} \cdot  \left(\frac{P_1^{H_y}}{P_2^{H_y}}\right)^{-1}\left (
		\frac{\partial E_z}{\partial x}
		\right)^{n+1} $.
	\end{enumerate}
	\begin{figure}[!t]
		\centering
		\includegraphics[scale=0.7]{figures/Ez.eps}
		\caption{
			The numerical domain for step 1.
			Squares denote boundary points. x denotes interior points of the numerical domain.  The points where $H_x$ and $H_y$ are evaluated (little blue and red dots resp.) are shown for reference.}
		\label{fig:E}
		
	\end{figure}
	
	As mentioned before, these  elliptic equations  can be solved  to fourth order   if the boundary conditions of the solutions and of the non-homogeneous term are known to fourth order \cite{singer_turkel}.
	Let us examine each step in the scheme separately:\\[1mm]
	{\bf Step 1 } (See Figure \ref{fig:E}):  	$\Gw^{E_z}=\{(x_i,y_j)\}_{i,j\in [N]}$;
% boundary conditions :  the boundary values of $\delta_{\tau} E_z$ are given explicitly (=0) from the boundary conditions $\vec{\nu}\times \vec{E}=0$; non-homogeneous term:
%	$\left (\frac{\partial H_y}{\partial x}-\frac{\partial H_x}{\partial y}\right )^{n+\frac{1}{2}}$ can be estimated using Pad\'e approximation (see  Appendix \ref{appendix}) at all points 
%	$(x_i,y_j)$
%	which do not belong to $\partial [0,1]^2$.  However, at these boundary points, the equations readily imply that  
%	$\left (\frac{\partial H_y}{\partial x}-\frac{\partial H_x}{\partial y}\right )^{n+\frac{1}{2}}=0$ on $\partial [0,1]^2$.
\\[1mm]
	{\bf Step 2} (See Figure \ref{fig:Hx}): 
 $\Gw^{H_x}=\{(x_i,y_{j+\frac{1}{2}})\}_{i\in [N],j\in [N-1]}$; 
	non-homogeneous term:  
	$\left (\frac{E_z}{\partial y}\right)^{n+1} $
	can be approximated  using Pad\'e approximation  in Appendix \ref{appendix} at all discretization points;
	boundary conditions:
	we  estimate 
	$\frac{H_x^{n+3/2}-H_x^{n+\frac{1}{2}}}{\Delta \tau}$ on the boundary as follows.
	The equation  
	$$
	\frac{\partial H_x}{\partial \tau}=-\frac{1}{Z}\frac{\partial E_z}{\partial y}
	$$
	implies that $H_x(0,y)$ and $H_x(1,y)$ do not depend on $\tau$, that is 
	$$\frac{H_x^{n+3/2}-H_x^{n+\frac{1}{2}}}{\Delta \tau}=0$$ if $x=0$ or $x=1$.
	On the remaining part of the boundary ($y=0,1$)  we use Taylor series:
	$$
	\frac{	H_x^{n+3/2}-H_x^{n+\frac{1}{2}}}{\Delta \tau}=
	\left(\frac{\partial H_x}{\partial \tau}\right)^{n+1}+
	\frac{\Delta \tau^2}{24}\left(\frac{\partial^3 H_x}{\partial \tau^3}\right)^{n+1}+O(\Delta \tau^4).
	$$
	%$$
	%=\left(\frac{\partial H_x}{\partial t}\right)^{n+1}\left(1+
	%\frac{\Delta t^2}{24}\cdot 
	%\left (\frac{\partial^2 H_x}{\partial t^2}\right)^{n+1}+
	%O(\Delta t^4)
	%\right)
	%$$
	Recall that
	$\left(\frac{\partial H_x}{\partial \tau}\right)^{n+1}=-\frac{1}{Z}\left(\frac{\partial E_z}{\partial y}\right)^{n+1}$ has been already approximated to fourth order. 
	If one uses just the approximation 
	$$
	\frac{	H_x^{n+3/2}-H_x^{n+\frac{1}{2}}}{\Delta \tau}=
	\left(\frac{\partial H_x}{\partial \tau}\right)^{n+1}
	$$
	then the boundary conditions are of order $h^4+\Delta \tau^2$.
	
	The remaining term 
	can be  approximated to order $\Delta \tau^2 h^2$ as follows.
	The equations imply that   $\frac{\partial^2E_z}{\partial \tau^2}=\Delta E_z$, and therefore
	$$
	\frac{\partial^3 H_x}{\partial \tau^3}=
	-\frac{1}{Z}\Delta \frac{\partial E_z }{\partial y}.
	$$
	The Laplacian of $\frac{\partial E_z }{\partial y}$ can be estimated at the boundary points  to order $h^2$ using central and one-sided schemes for second-order derivatives.
	In such a case the boundary conditions are of order $h^4+\Delta \tau^2h^2.$
	The numerical simulations   presented in  Table \ref{table:conv_rates} show the affect of the boundary conditions accuracy on the convergence rates.\\[1mm]
	{\bf Step 3} : \\ $\Gw^{H_y}=\{(x_{i+\frac{1}{2}},y_{j})\}_{i\in [N-1],j\in [N]}$; the boundary terms and the boundary conditions are treated by a similar way to step 2.
	
	\begin{figure}[!t]
		\centering
		\begin{subfigure}[h]{0.4\textwidth}
			\includegraphics[scale=0.45]{figures/Hx.eps}
		\end{subfigure}
		\begin{subfigure}[h]{0.4\textwidth}
			\includegraphics[scale=0.45]{figures/Hy.eps}
		\end{subfigure}
		\caption{The numerical domains for steps 2,3.
			Squares denote $\partial \Gw^{H_x}$ (left)  and $\partial \Gw^{H_y}$. "x" denotes $\intr{\Gw^{H_x}}$ (left) and $\intr{\Gw^{H_y}}$ . $\Gw^{E_z}$ (little black dots) is shown for reference.}
		\label{fig:Hx}			\end{figure}
	\subsection{Numerical simulations}
	We test analytical solutions in the case $Z=1$,  
	$(k_x,k_y)=(4,3)$ and $\omega:=\sqrt{k_x^2+k_y^2}$.
	Namely, 
	\begin{align*}
		&
		E_z=\cos(\omega \tau)
		\sin{\left( \pi  k_x x\right) } \sin{\left( \pi  k_y y\right) }
		\\ &
		H_x=-\frac{\sin( \omega \tau)}{\omega}
		\pi k_y \sin( \pi  k_x x)
		\cos{\left( \pi  k_y y\right) } \\&
		H_y=\frac{\sin\left( \omega \tau\right) }{\omega} \pi  k_x \cos{\left( \pi  k_x x\right) } \sin{\left( \pi  k_y y\right) }. \\&
	\end{align*}
	
	
	We use the mean absolute error 
	$$
	\frac{1}{N_{\tau}\cdot N^2}\sum_{i,j,n=0}^{N,N,N_{\tau}}\left | F^n_{\mathrm{numer.}}(i,j)-F_{\mathrm{true}}^n(i,j) \right|
	$$
	$(\vec{F}=(E_z,H_x,H_y))$ to calculate the (log-)convergence rates of our schemes.
	%	\begin{table} [h!]
		%	\centering
		%	\begin{tabular}{|p{0.7cm}|p{0.7cm}|p{2cm}|p{2cm}| p{2cm}|}
			%		\hline
			%$h$ & $\mathrm{CFL}$ &$\Delta \tau^2+h^4$&$\Delta \tau^2h^2+h^4$& analytic
			%		\\ [0.5ex] 
			%		\hline 
			%$\frac{1}{20}$ & 0.5& & & \\
			%$\frac{1}{25}$ & 0.5 &3.74 &5.51&5.27\\
			%$\frac{1}{30}$ & 0.5 & 3.50 &5.55& 5.38 \\
			%$\frac{1}{35}$ & 0.5 &3.38 & 5.57&5.30\\
			%$\frac{1}{40}$ & 0.5& 3.31 &5.55 &5.22\\
			%\hline 
			%	\end{tabular}
		%	\caption{ Converge rates with several types of boundary conditions for $\delta_\tau \vec{H}$ in steps 2 and 3:	 
			%	Here $\Delta \tau=\frac{h}{2}$ where $h=\frac{1}{20},\frac{1}{25},\frac{1}{30},
			%	\frac{1}{35}, \frac{1}{40}
			%	$
			%	and final time $T=1$.}
		%	\label{table:conv_rates}
		
		
		%\end{table}
		\begin{table} [h!]
			\centering
			\begin{tabular}{|p{0.7cm}|p{0.7cm}|p{2cm}|p{2cm}|}
				\hline
				$h$ & $\mathrm{CFL}$ &$\Delta \tau^2+h^4$&$\Delta \tau^2h^2+h^4$
				\\ [0.5ex] 
				\hline 
				$\frac{1}{16}$ & 0.5& &  \\[0.5mm]
				$\frac{1}{32}$ & 0.5 &3.70 &5.45\\[0.5mm]
				$\frac{1}{64}$ & 0.5 & 3.21 &5.23 \\[0.5mm]
				$\frac{1}{128}$ & 0.5 &3.03 & 3.98\\[0.5mm]
				$\frac{1}{256}$ & 0.5& 2.99 &3.93 \\[0.5mm]
				\hline 
			\end{tabular}
			\caption{ Converge rates with several types of boundary conditions for $\delta_\tau \vec{H}$ in steps 2 and 3:	 
				Here $\Delta \tau=\frac{h}{2}$ where $h=\frac{1}{16},\frac{1}{32},\frac{1}{64},
				\frac{1}{128}, \frac{1}{256}
				$
				and final time $T=1$.}
			\label{table:conv_rates}
			
			
		\end{table}
		\begin{table} [h!]
			\centering
			\begin{tabular}{|p{0.7cm}|p{0.7cm}|p{2cm}| |p{4cm}|}
				\hline
				$h$ & $\mathrm{CFL}$ &error& convergence rate
				\\ [0.5ex] 
				\hline 
				$\frac{1}{16}$ & 0.5& 2.9e-3&  \\[0.5mm]
				$\frac{1}{32}$ & 0.5 &6.7e-5 &5.45\\[0.5mm]
				$\frac{1}{64}$ & 0.5 & 1.8e-6 &5.23 \\[0.5mm]
				$\frac{1}{128}$ & 0.5 &1.1e-7 & 3.98\\[0.5mm]
				$\frac{1}{256}$ & 0.5& 7.5e-9 &3.93 \\[0.5mm]
				\hline \hline
				$\frac{1}{16}$ & 0.5& 0.38&  \\[0.5mm]
				$\frac{1}{32}$ & 0.5 &0.22 &0.75\\[0.5mm]
				$\frac{1}{64}$ & 0.5 & 5.2e-3 &5.4 \\[0.5mm]
				$\frac{1}{128}$ & 0.5 &2.2e-4 & 4.52\\[0.5mm]
				$\frac{1}{256}$ & 0.5& 1.6e-5 &3.822 \\[0.5mm]
				\hline
			\end{tabular}
			\caption{ Converge rates with $k_x=4,k_y=3$ (upper table)
				and  $k_x=17,k_y=18$.
				Here $\Delta \tau=\frac{h}{2}$ where $h=\frac{1}{16},\frac{1}{32},\frac{1}{64},
				\frac{1}{128}, \frac{1}{256}
				$
				and final time $T=1$.}
			\label{table:conv_rates_2}
			
			
		\end{table}
		\subsection{Comparison with non-compact fourth order scheme and with data-driven scheme (AI(2,3)) trained on high wave-numbers}
		\begin{table} [h!]
			\centering
			\begin{tabular}{|p{0.7cm}|p{0.7cm}|p{4cm}|p{4cm}|p{4cm}|}
				\hline
				$h$ & $\mathrm{CFL}$ &compact-4th& AI(2,3)&non-compact-4th
				\\ [0.5ex] 
				\hline 
				$\frac{1}{16}$ & 0.5& 6.9e-5&  6e-5&8.7e-5 \\[0.5mm]
				$\frac{1}{32}$ & 0.5 &1.5e-6 &6.9e-6&1.18e-5 \\[0.5mm]
				$\frac{1}{64}$ & 0.5 & 6.8e-8 &8.2e-7& 1.5e-6 \\[0.5mm]
				$\frac{1}{128}$ & 0.5 &4.9e-9 & 1e-7&1.9e-7 \\[0.5mm]
				$\frac{1}{256}$ & 0.5& 3.7e-10 &1.2e-8&2.4e-8 \\[0.5mm]
				\hline \hline
				$\frac{1}{16}$ & 0.5& 2.9e-3&  2.6e-4&8.1e-4 \\[0.5mm]
				$\frac{1}{32}$ & 0.5 &6.7e-5 &2.2e-5&1.1e-4 \\[0.5mm]
				$\frac{1}{64}$ & 0.5 & 1.8e-6 &1.8e-6& 1.4e-5 \\[0.5mm]
				$\frac{1}{128}$ & 0.5 &1.1e-7 & 1.8e-7&1.8e-6 \\[0.5mm]
				$\frac{1}{256}$ & 0.5& 7.5e-9 &1.9e-8&2.2e-7 \\[0.5mm]
				\hline \hline
				$\frac{1}{16}$ & 0.5& 0.38& 3.4e-3 &4.4e-3\\[0.5mm]
				$\frac{1}{32}$ & 0.5 &0.22 &2.9e-3&3.4e-3 \\[0.5mm]
				$\frac{1}{64}$ & 0.5 & 5.2e-3 &1e-4& 1.7e-3\\[0.5mm]
				$\frac{1}{128}$ & 0.5 &2.2e-4 & 2.84e-6&2.8e-4\\[0.5mm]
				$\frac{1}{256}$ & 0.5& 1.6e-5 &1.02e-6& 3.5e-5\\[0.5mm]
				\hline
			\end{tabular}
			\caption{ Errors for solutions  with wave-numbers $k_x=1,k_y=2$ (upper table), $k_x=4,k_y=3$ (middle table),
				and  $k_x=17,k_y=18$.
				Here $\Delta \tau=\frac{h}{2}$ where $h=\frac{1}{16},\frac{1}{32},\frac{1}{64},
				\frac{1}{128}, \frac{1}{256}
				$
				and final time $T=1$.}
			\label{table:conv_rates_3}
		\end{table}

		%\section{Convergence rates with boundary conditions for $H$ of order 
			%	$dt^2+h^4$}
		%We test analytical solutions in the case $c=1$,  such that 
		%$(k_x,k_y)=(4,3)$ and $\Omega=\sqrt{k_x^2+k_y^2}$.
		%We take $dt=\frac{h}{2}$ where $h=\frac{1}{20},\frac{1}{25},\frac{1}{30},
		%\frac{1}{35}, \frac{1}{40}$ from time 0 to time 1.
		%We consider the mean absolute error, that is 
		%$$
		%\frac{1}{N_t\cdot N^2}\sum_{i,j,n=0}^{N,N,N_t}\left | F^n(i,j)-F_{\mathrm{true}}^n(i,j) \right|
		%$$
		%$(\vec{F}=(E_z,H_x,H_y))$
		%to calculate the convergence rates as a function of $1/h$:
		%$$
		%(3.02, 3.15, 3.14,  3.15)
		%$$
		
		%	\begin{table} [h!]
			%	\centering
			%	\begin{tabular}{|p{0.7cm}|p{0.7cm}|p{0.7cm}| p{5cm}|p{5cm}|}
				%		\hline
				%		$N$& $N_t$ & $T$ &b.c for $H$: $O(dt^2+h^2)$& 
				%		b.c for $H$: $O(dt^2+h^4)$\\
				%		
				%		
				%
				%		
				%		
				%		
				%		\hline
				%	\end{tabular}
			%	\caption{Convergence rates
				%	}
			%	\label{table:tab3}
			%\end{table}
			\appendix
			\section{}\label{appendix}
			Assume that $f(x)$ is known at N points $x_0,x_1, ..,x_{N-1}$
			and we want to estimate to fourth order $f'(x)$ at N-1 points
			$x_{\frac{1}{2}},..,x_{N-3/2}$. This can be done  by solving the system of equations
			\begin{equation}\label{eq:lhs}
				\frac{1}{24}
				\begin{pmatrix}
					26     & -5     & 4      & -1     & 0      & \dots  & 0      \\
					1      & 22     & 1      & 0      & 0      & \dots  & 0      \\
					0      & 1      & 22     & 1      & 0      & \dots  & 0      \\
					\vdots & \ddots & \ddots & \ddots & \ddots & \ddots & \vdots \\
					0      & 0      & \dots  & 1      & 22     & 1      & 0      \\
					0      & 0      & \dots  & 0      & 1      & 22     & 1      \\
					0      & 0      & \dots  & -1     & 4      & -5     & 26
				\end{pmatrix}
				\begin{pmatrix}
					f'(x_{\frac{1}{2}}) \\
					f'(x_{3/2}) \\
					\cdots \\
					\cdots\\
					\cdots \\
					\cdots \\
					f'(x_{N-3/2})
				\end{pmatrix}=
				\frac{1}{\Delta x}
				\begin{pmatrix}
					f(x_{1})-f(x_0)  \\
					\cdots \\
					\cdots\\
					\cdots \\
					\cdots \\
					\cdots \\
					f(x_{N-1)}-f(x_{N-2})
				\end{pmatrix}.
			\end{equation}
		The matrix operator in the left hand side of \eqref{eq:lhs} will be denoted by $a$.
%		If one wants a symmetric operator, then one can redefine the system with compromising the accuracy on the boundary points.
%		\begin{equation}\label{eq:lhs2}
%			\frac{1}{24}
%			\begin{pmatrix}
%				24    & 0    & 0      & 0     & 0      & \dots  & 0      \\
%				1      & 22     & 1      & 0      & 0      & \dots  & 0      \\
%				0      & 1      & 22     & 1      & 0      & \dots  & 0      \\
%				\vdots & \ddots & \ddots & \ddots & \ddots & \ddots & \vdots \\
%				0      & 0      & \dots  & 1      & 22     & 1      & 0      \\
%				0      & 0      & \dots  & 0      & 1      & 22     & 1      \\
%				0      & 0      & \dots  & 0     & 0      & 0     & 24
%			\end{pmatrix}
%			\begin{pmatrix}
%				f'(x_{\frac{1}{2}}) \\
%				f'(x_{3/2}) \\
%				\cdots \\
%				\cdots\\
%				\cdots \\
%				\cdots \\
%				f'(x_{N-3/2})
%			\end{pmatrix}=
%			\frac{1}{\Delta x}
%			\begin{pmatrix}
%				f(x_{1})-f(x_0)  \\
%				\cdots \\
%				\cdots\\
%				\cdots \\
%				\cdots \\
%				\cdots \\
%				f(x_{N-1)}-f(x_{N-2})
%			\end{pmatrix}.
%		\end{equation}

			%By Taylor approximation:
			%$$
			%\frac{f(x+h/2)-f(x-h/2)}{h}=f'(x)+\frac{f'''(x)h^2}{24}+O(h^4)
			%$$
			%$$=
			%f'(x)+\frac{1}{24}(f'(x+h)-2f'(x+h)+f'(x-h))+O(h^4)
			%$$
			\section{}\label{appendxib}
			In the current appendix we develop the compact numerical scheme.
			We do so in a more general case  for which $\varepsilon$ and $\mu$ are not necessarily constants.
			 
				We first achieve fourth order accuracy in time using a Taylor expansion:
			\begin{align*}
				\left (\frac{\partial D_z}{\partial t} \right)^{n+\frac{1}{2}} &= \frac{D_z^{n+1} - D_z^{n}}{\Delta t} - \frac{(\Delta t^2)}{24} \frac{\partial D^3_z}{\partial t^3} + O(\Delta t)^4 \\
				&= \frac{D_z^{n+1} - D_z^{n}}{\Delta t} - \frac{(\Delta t^2)}{24} \partial_{tt} \left(\frac{\partial H_y}{\partial x}
				- \frac{\partial H_x}{\partial y} -  J_z \right)
				+ O(\Delta t)^4
			\end{align*}
			We now use the relation $H \!=\! \frac{1}{\mu} B$. For simplicity we define
			\begin{equation}
				\label{Dhat}
				\delta_t D_z: = \frac{D_z^{n+1} - D_z^{n}}{\Delta t} .
			\end{equation}
			We then get
			\begin{align*}
				\left (\frac{\partial D_z}{\partial t} \right)^{n+\frac{1}{2}}  &= \delta_t D_z - \frac{\Delta t^2}{24} \partial_{tt} \left(\frac{\partial {B_y \over \mu}}{\partial x }
				- \frac{\partial{B_x \over \mu}}{\partial y} - J_z \right) + O(\Delta t)^4 \\
				&= \delta_t D_z - \frac{\Delta t^2}{24} \frac{\partial}{\partial t} \left(\frac{\partial^2 {B_y \over \mu}}{\partial x \partial t}
				- \frac{\partial^2 {B_x \over \mu}}{\partial y \partial t} - \frac{\partial J_z}{\partial t} \right) + O(\Delta t)^4.
			\end{align*}
			Recall that 
			
			$$
			\frac{\partial B_y}{\partial t}=\frac{\partial E_z}{\partial x}-\frac{\partial E_x}{\partial z},\qquad
			\frac{\partial B_x}{\partial t}=\frac{\partial E_y}{\partial z}-\frac{\partial E_z}{\partial y}
			$$
			and since $\mu$ does not depend on $t$,
			$$
			\frac{\partial (B_y/\mu)}{\partial t}=
			\frac{1}{\mu}\left (
			\frac{\partial E_z}{\partial x}-\frac{\partial E_x}{\partial z}
			\right), \qquad
			\frac{\partial (B_x/\mu)}{\partial t}=
			\frac{1}{\mu}\left (
			\frac{\partial E_y}{\partial z}-\frac{\partial E_z}{\partial y}
			\right) .
			$$
			
			Hence
			$$
			\left(\frac{\partial^2 {B_y \over \mu}}{\partial x \partial t}
			- \frac{\partial^2 {B_x \over \mu}}{\partial y \partial t}  \right)=
			\left (
			\frac{\partial}{\partial x} \frac{1}{\mu}\left (
			\frac{\partial E_z}{\partial x}-\frac{\partial E_x}{\partial z}
			\right)-\frac{\partial }{\partial y} \frac{1}{\mu}\left (
			\frac{\partial E_y}{\partial z}-\frac{\partial E_z}{\partial y}
			\right) 
			\right)
			$$
			\begin{equation}\label{eq:mu_diff}
				=\frac{1}{\mu}
				\left(
				\frac{\partial^2 E_z}{\partial x^2}
				+\frac{\partial^2 E_z}{\partial y^2}
				-\frac{\partial^2 E_x}{\partial x\partial z}
				-\frac{\partial^2 E_y}{\partial y\partial z}
				\right)+
				\frac{\partial (1/\mu)}{\partial x}\left ( \frac{\partial E_z}{\partial x}-\frac{\partial E_x}{\partial z}\right)
				-
				\frac{\partial (1/\mu)}{\partial y}\left ( \frac{\partial E_y}{\partial z}-\frac{\partial E_z}{\partial y}\right).
			\end{equation}
			After adding and subtracting $\frac{\partial^2 E_z  }{\partial z^2}$
			in the first parenthesis in \eqref{eq:mu_diff}, and adding and subtracting
			$\frac{\partial (1/\mu)}{\partial z}\frac{\partial E_z}{\partial z}$
			we obtain that  \eqref{eq:mu_diff} equals to
			$$
			div(\mu^{-1}\nabla E_z)-div(\mu^{-1}\frac{\partial \vec{E}}{\partial z}).
			$$
			This yields
			\begin{align*}
				&
				\left ( \frac{\partial H_y}{\partial x}-
				\frac{\partial H_x}{\partial y}-J_z
				\right)^{n+\frac{1}{2}}=\\&
				\frac{\partial D_z^{n+\frac{1}{2}}}{\partial t} =
				\delta_t D_z - \frac{\Delta t^2}{24} \frac{\partial}{\partial t}
				\left(
				div(\mu^{-1}\nabla E_z)-div(\mu^{-1}\frac{\partial \vec{E}}{\partial z})
				-\frac{\partial J_z}{\partial t}
				\right)+O(\Delta t	)^4= \\&
				\varepsilon \delta_t  E_z - \frac{\Delta t^2}{24}
				\left(
				div(\mu^{-1} \nabla \delta_t  E_z)
				- div\left (\mu^{-1}\frac{\partial ^2\vec{E}}{\partial t \partial z}
				\right)
				-\frac{\partial^2 J_z}{\partial t^2}
				\right)
				+O(\Delta t	)^4.
			\end{align*}
			The term 
			$$
			\left ( \frac{\partial H_y}{\partial x}-
			\frac{\partial H_x}{\partial y}-J_z
			\right)^{n+\frac{1}{2}}
			$$
			can be approximated to fourth order using the Pade approximation given in Appendix \ref{appendix} \cite{yefet_turkel}(see Section \ref{sec:TE}). 
			The term
			$$
			div\left(\mu^{-1}\frac{\partial^2\vec{E}}{\partial t \partial z}\right )^{n+\frac{1}{2}}
			=div\left(
			\mu^{-1}\frac{\partial }{ \partial z}\frac{1}{\varepsilon}(
			\nabla \times \vec{H} -\vec{J}
			)
			\right)^{n+\frac{1}{2}}
			$$
			can be estimated from the previous
			time  step to the second order  accuracy in $h$ since it is already multiplied by $\Delta t ^2$. 
			In fact, since $div(\nabla\times H)=0$ we have to approximate to second order only first and second derivatives  of $\vec{H}$.
			Hence,
			$E_z^{n+1}=E^n+\Delta t\cdot \delta_t E_z$ where 
			$\delta_t E_z$ solves the non-negative symmetric  elliptic equation
			\begin{equation}\label{eq:elliptic_div}
				\varepsilon \delta_t E_z-\frac{\Delta t^2}{24}div(\mu^{-1}\nabla \delta_t E_z)=F^{n+\frac{1}{2}},
			\end{equation}
			where 
			$$
			F^{n+\frac{1}{2}}=
			\left ( \frac{\partial H_y}{\partial x}-
			\frac{\partial H_x}{\partial y}-J_z
			\right)^{n+\frac{1}{2}}+\frac{\Delta t^2}{24}\left (
			div\left(
			\mu^{-1}\frac{\partial }{ \partial z}\frac{1}{\varepsilon}(
			\nabla \times \vec{H} -\vec{J}
			)
			\right )-\frac{\partial^2J_z}{\partial t^2}
			\right)^{n+\frac{1}{2}}.
			$$
			For fourth order compact schemes for \eqref{eq:elliptic_div} see \cite{britt_tsy_tur}. 
				\subsection{Boundary conditions for $\delta_t \vec{E}$ and  $\delta_t \vec{H}$}
			In order to solve \eqref{eq:elliptic_div} to fourth order  one has to supply certain (Dirichlet) boundary conditions for $\delta_t \vec{E}$ and  $\delta_t \vec{H}$ to fourth order.
			We will show the construction in the first  time step. The next steps  will then follow by the staggered structure.
			We assume that $E^0$ and $H^{\frac{1}{2}}$ are given to fourth order.
			For $H^{\frac{1}{2}}$ this can be done using  standard Taylor approximation.
			Next,
			$$
			\frac{\vec{E}^{1}-\vec{E}^{0}}{\Delta t}=\frac{\partial \vec{E}^{\frac{1}{2}}}{\partial t}+
			\Delta t^2\frac{\partial^3 \vec{E}^{\frac{1}{2}}}{\partial t^3}+O(\Delta t^4).
			$$
			The equation for $\vec{E}$ imply that 
			$$
			\frac{\partial \vec{E}^{\frac{1}{2}}}{\partial t}=\varepsilon^{-1}\cdot \nabla \times \vec{H}^{\frac{1}{2}}.
			$$
			The latter term is known to fourth order  and therefore can be approximated using Pad'e scheme in Appendix \ref{appendix} to fourth order. 
			Next we consider the term 
			\begin{align*}
				&
				\frac{\partial^3 \vec{E}^{\frac{1}{2}}}{\partial t^3} =
				\varepsilon^{-1}\nabla \times  \frac{\partial^2 H^{\frac{1}{2}}}{\partial t^2}=
				\\&
				-\varepsilon^{-1}\nabla\times \mu^{-1}
				\nabla \times \frac{\partial \vec{E}^{\frac{1}{2}}}{\partial t}=
				-\varepsilon^{-1}\nabla\times \mu^{-1} \nabla \times
				\varepsilon^{-1}\nabla\times \vec{H}^{\frac{1}{2}}.
			\end{align*}
			The term $ \nabla\times \vec{H}^{\frac{1}{2}}$ is known to fourth order and therefore it is sufficient to approximate its  second-derivatives to second order. Since 
			$\frac{\partial^3 \vec{E}^{\frac{1}{2}}}{\partial t^3}$ is multiplied by $\Delta t^2$ we have accomplished our estimation to order $h^4+\Delta t^2h^2$.
			We demonstrate the above construction   in the following section.
			
			
			\begin{center}
				{\bf Acknowledgments}
			\end{center}
			\begin{thebibliography}{99}
				
				
				\bibitem{britt_tsy_tur}					
				Britt, Steven, Semyon Tsynkov, and Eli Turkel. "Numerical simulation of time-harmonic waves in inhomogeneous media using compact high order schemes." Communications in Computational Physics 9.3 (2011): 520-541.					
				\bibitem{Morton}
				Morton,K.W.,Mayers,D.F.:NumericalSolutionofPartialDifferentialEquations,2ndedn.Cambridge
				University Press, (2005)
				\bibitem{chen}
				Chen, Wenbin, Xingjie Li, and Dong Liang. "Energy-conserved splitting FDTD methods for Maxwell?s equations." Numerische Mathematik 108.3 (2008): 445-485.
				
				\bibitem{rolf}
				Leis, Rolf. Initial boundary value problems in mathematical physics. Courier Corporation, 2013.
				\bibitem{sakka}
				Sakkaplangkul, Puttha, and V. Bokil. "CONVERGENCE ANALYSIS OF YEE-FDTD SCHEMES FOR 3D MAXWELL?S EQUATIONS IN LINEAR DISPERSIVE MEDIA." International journal of numerical analysis and modeling 18.4 (2021).	
				\bibitem{singer_turkel}
				I. ~Singer, and E.~ Turkel. "High-order finite difference methods for the Helmholtz equation." Computer methods in applied mechanics and engineering 163.1-4 (1998): 343-358.					
				\bibitem{yee}
				K.~Yee, Numerical solution of initial boundary value problems involving Maxwell's equations in isotropic media, {\em IEEE Transactions on antennas and propagation} {\bf 14.3} (1966), 302-307.
				\bibitem{yefet_turkel}
				A.~Yefet and E.~Turkel,
				Fourth Order Compact Implicit Method for the Maxwell Equations with Discontinuous Coefficients
				{\em Applied Numerical Mathematics}  {\bf 33} (2000), 125--134.
			\end{thebibliography}
		\end{document}